\documentclass[10pt]{article}
% \usepackage{geometry}
% \geometry{margin=0.2in}
\usepackage[utf8]{inputenc}

\nonstopmode
% \usepackage{minted}[cache=false]
\usepackage{graphicx} % Required for including pictures
\usepackage[figurename=Figure]{caption}
% \usepackage{float}    % For tables and other floats
\usepackage{amsmath}  % For math
\usepackage{amssymb}  % For more math
\usepackage{fullpage} % Set margins and place page numbers at bottom center
% \usepackage{paralist} % paragraph spacing
% \usepackage{subfig}   % For subfigures
%\usepackage{physics}  % for simplified dv, and 
% \usepackage{enumitem} % useful for itemization
% \usepackage{siunitx}  % standardization of si units
\usepackage{hyperref}
% \usepackage{mmacells}
% \usepackage{listings}
% \usepackage{svg}
% \usepackage{xcolor, soul}
\usepackage{bm}
\usepackage{braket}
% \usepackage{cancel}
% \usepackage{setspace}
% \usepackage{listings}
% \usepackage{listings}
% \usepackage[autoload=true]{jlcode}
% \usepackage{pygmentize}

% \definecolor{cambridgeblue}{rgb}{0.64, 0.76, 0.68}

% \sethlcolor{cambridgeblue}

\usepackage[margin=1.8cm]{geometry}
\newcommand{\C}{\mathbb C}
\newcommand{\D}{\bm D}
\newcommand{\R}{\mathbb R}
\newcommand{\Q}{\mathbb Q}
\newcommand{\Z}{\mathbb Z}
\newcommand{\N}{\mathbb N}
\newcommand{\PP}{\mathbb P}
\newcommand{\A}{\mathbb A}
\newcommand{\F}{\mathbb F}
\newcommand{\1}{\mathbf 1}
\newcommand{\ip}[1]{\left< #1 \right>}
\newcommand{\abs}[1]{\left| #1 \right|}
\newcommand{\norm}[1]{\left\| #1 \right\|}

\def\Tr{{\rm Tr}}
\def\tr{{\rm tr}}
\def\Var{{\rm Var}}
\def\calA{{\mathcal A}}
\def\calB{{\mathcal B}}
\def\calD{{\mathcal D}}
\def\calE{{\mathcal E}}
\def\calG{{\mathcal G}}
\def\from{{:}}
\def\lspan{{\rm span}}
\def\lrank{{\rm rank}}
\def\bd{{\rm bd}}
\def\acc{{\rm acc}}
\def\cl{{\rm cl}}
\def\sint{{\rm int}}
\def\ext{{\rm ext}}
\def\lnullity{{\rm nullity}}
% \DeclareSIUnit\clight{\text{\ensuremath{c}}}
% \DeclareSIUnit\fm{\femto\m}
% \DeclareSIUnit\hplanck{\text{\ensuremath{h}}}


% \lstdefinelanguage{julia}%
%   {morekeywords={abstract,break,case,catch,const,continue,do,else,elseif,%
%       end,export,false,for,function,immutable,import,importall,if,in,%
%       macro,module,otherwise,quote,return,switch,true,try,type,typealias,%
%       using,while},%
%    sensitive=true,%
% %    alsoother={$},%
%    morecomment=[l]\#,%
%    morecomment=[n]{\#=}{=\#},%
%    morestring=[s]{"}{"},%
%    morestring=[m]{'}{'},%
% }[keywords,comments,strings]%

% \lstset{%
%     language         = Julia,
%     basicstyle       = \ttfamily,
%     keywordstyle     = \bfseries\color{blue},
%     stringstyle      = \color{magenta},
%     commentstyle     = \color{ForestGreen},
%     showstringspaces = false,
% }

% $
\begin{document}
\begin{center}
	\hrule
	\vspace{.4cm}
	{\textbf { \large CAS PY 452 --- Quantum Physics II}}
\end{center}
Emmy Blumenthal \hspace{\fill} \hspace{\fill}  \textbf{} Discussion Notes\  \\
\textbf{Date:}\  Nov 30, 2022   \hspace{\fill} \textbf{Email:}\ emmyb320@bu.edu 
\vspace{.4cm}
\hrule

\section*{Time-dependent perturbation theory for the two-level system}

\paragraph{Problem:}

Consider a two-level system with energy eigenstates $\{\ket{0},\ket{1}\}$ and energy eigenvalues $\{E_0,E_1\}$.
That is, it's Hamiltonian in the basis $\{\ket{0},\ket{1}\}$ is,
\begin{align}
	\hat H
	=
	\left(
		\begin{array}[]{cc}
			H_{00} & H_{01}\\
			H_{10} & H_{11}\\
		\end{array}
	\right)
	=
	\left(
		\begin{array}[]{cc}
			E_0 & 0\\
			0 & E_1\\
		\end{array}
	\right).
\end{align}
The system is then subject to a time-dependent perturbation $\hat V(t)$ which couples the unperturbed states:
\begin{align}
	\hat V(t)
	=
	\left(\begin{cases}
		\frac{1}{2}\alpha \eta e^{\eta t} & t <0\\
		\frac{1}{2}\alpha \eta e^{-\eta t} & t \geq 0\\
	\end{cases}
	\right)
	\left(
		\begin{array}[]{cc}
			0 & 1\\
			1 & 0\\
		\end{array}
	\right)
	=
	\frac{\alpha}{2}\eta e^{-\eta|t|}
	\left(
		\begin{array}[]{cc}
			0 & 1\\
			1 & 0\\
		\end{array}
	\right)
\end{align}
Assume that $\eta>0$, $\alpha \ll \hbar$ and $\hbar \eta \ll E_0,E_1$ so that we can apply first-order time-dependent perturbation theory.
Given that the system is in state $\ket{0}$ at $t \to -\infty$, find the probability that the state will be in state $\ket{1}$ at $t \to \infty$ using time-dependent perturbation theory.
Analyze these results as $\eta \to 0$ and $\eta \to \infty$.
What type of perturbation does $\hat V(t)$ look like as $\eta \to \infty$?
% Also consider the case where $\alpha = \beta/\eta$.

\paragraph{Reminder:}

In time-dependent perturbation theory, we attempt to solve $i\hbar \partial_t \ket{\psi(t)} = (\hat H + \hat V(t))\ket{\psi(t)}$.
We find a systematic method by moving to the interaction representation:
\begin{align}
	\ket{\psi(t)}_I 
	=
	e^{i \hat H t/\hbar}\ket{\psi(t)},
	\qquad
	\hat V_I(t)
	=
	e^{i \hat H t/\hbar} \hat V(t)
	e^{-i \hat H t/\hbar},
\end{align}
so the TDSE becomes,
\begin{align}
	i \hbar \partial_t \ket{\psi(t)}_I 
	=
	\hat V_I(t) \ket{\psi(t)}_I.
\end{align}
If we write $\ket{\psi(t)}_I = \sum_n c_n(t) \ket{n}$, where $\ket{n}$ is the $n$th stationary state, then 
$\bra{m} i \hbar \partial_t \ket{\psi(t)}_I = \bra{m} \hat V_I(t) \ket{\psi(t)}_I$, becomes,
\begin{align}
	i \hbar \partial_t c_m(t)
	=
	\sum_{n} 
	V_{mn}(t) e^{i \omega_{mn} t} c_n(t),
\end{align}
where $V_{mn}(t) = \braket{m | \hat V(t)|n}$ and $\omega_{mn} = (E_m - E_n)/\hbar$.
If a system begins in state $\ket{i}$, we can form a perturbative approximation by solving iteratively.
This approximation gives $c_n(t) = c_n^{(0)} +c_n^{(1)} +c_n^{(2)} +\cdots$, where,
\begin{align}
	c_n^{(0)}(t)
	&=
	\delta_{ni},
	\\
	c_n^{(1)}(t)
	&=
	-\frac{i}{\hbar}
	\int_{t_0}^t dt' e^{i \omega_{ni} t'} V_{ni}(t')
	\\
	c_n^{(2)}(t)
	&=
	\left(-\frac{i}{\hbar}\right)
	\sum_{m}
	\int_{t_0}^t
	dt'
	e^{i \omega_{nm}t'}
	V_{nm} (t')
	\left(-\frac{i}{\hbar}\right)
	\int_{t_0}^{t'}
	dt''
	e^{i \omega_{mi}t''}
	V_{mi}(t'').
\end{align}
This is called the Dyson series.\footnote{
	The Dyson series is related to Feynman diagrams. If you're curious about this connection, consider reading these notes (\url{https://web2.ph.utexas.edu/~vadim/Classes/2022f/dyson.pdf}) from a class at UT Austin.
}

\paragraph{Solution:}

To compute the transition probability, we compute the integral:
\begin{align}
	c_{1}^{(1)}(t \to \infty)
	&=\nonumber
	-\frac{i}{\hbar}
	\int_{-\infty}^\infty
	dt'
	e^{i \omega_{10} t'}
	\braket{1 | \hat V(t) | 0}
	=
	-\frac{i\alpha\eta}{2\hbar}
	\int_{-\infty}^\infty
	dt'
	e^{i \omega_{10} t'}
	e^{-\eta |t|}
	\\
	&
	=
	-
	\frac{i\alpha\eta}{2\hbar}
	\left(
		\int_{-\infty}^0 dt e^{(i \omega_{10} + \eta t)t}
		+
		\int_{0}^{\infty} dt e^{(i \omega_{10} - \eta t)t}
	\right)
	=
	-
	\frac{i\alpha\eta}{2\hbar}
	\left(
		\int_{0}^\infty dt e^{-(i \omega_{10} + \eta )t}
		+
		\int_{0}^{\infty} dt e^{-(-i \omega_{10} + \eta )t}
	\right)
	\nonumber
	\\
	&
	=
	-
	\frac{i\alpha\eta}{2\hbar}
	\left(
		\frac{1}{\eta + i \omega_{10}}
		+
		\frac{1}{\eta - i \omega_{10}}
	\right)
	=
	-
	\frac{i\alpha}{\hbar}
	\left(
		\frac{\eta^2}{\eta^2 + \omega_{10}^2}
	\right).
\end{align}
This means the probability of transition is,
\begin{align}
	p_{0 \to 1}
	=
	|c_1^{(1)}(t \to \infty)|^2
	=
	\frac{\alpha^2}{\hbar^2}
	\left(
		\frac{\eta^2}{\eta^2 + \omega_{10}^2}
	\right)^2.
\end{align}
We can see that as $\eta \to 0$, the probability of transition goes to zero: $c_1^{(1)} (t \to \infty)\to 0$.
When $\eta \to \infty$, $\alpha \eta e^{-\eta|t|}/2 \to \alpha \delta(t)$, so we have a delta-function perturbation at $t=0$.
The integral $\alpha\int_{-\infty}^\infty \delta(t) e^{i \omega_{10}t} dt = \alpha$ is consistent with $c_1^{(1)} (t \to \infty) = -i\alpha/\hbar$ as $\eta \to \infty$.

% Notice that from the above, we can see that $c_1^{(1)}(0) =
% -\frac{i\alpha\eta}{2\hbar}
% \left(
% 	\frac{1}{\eta+i \omega_{10}}
% \right)
% =
% -
% \frac{\alpha\eta}{2 \hbar \omega_{10}}
% +
% O(\eta^2).
% $

% From the above result, we can easily see that $c_1^{(1)} ( 0) = $

% Now, let $\alpha= \beta/\eta$, so $c_1^{(1)}(t \to \infty) = -(i\beta/\hbar)(\eta/(\eta^2 + \omega_{10}^2))$, so as $\eta \to 0$, $c_1^{(1)}(t \to \infty) \to 0$.
% However, looking at our earlier results, we can see that $\lim_{\eta \to 0}c_1^{(1)} (0) = -\frac{i\beta}{2\hbar}\left(
% 	\frac{1}{0+ i \omega_{10}}
% \right)
% \ne 0.
% $
% This is an example of the adiabatic theorem.


% Now, let $\alpha = \beta/\eta$ and let $\xi = 1/\eta$, so,
% \begin{align}
% 	c_1^{(1)}(t \to \infty)
% 	&=
% 	-\frac{i\beta}{\hbar}
% 	\frac{  \xi }{\xi ^2 \omega_{10} ^2+1}
% 	=
% 	-\frac{i\beta}{\hbar}
% 	\xi 
% 	\frac{1}{1-(-\xi ^2 \omega_{10} ^2)}
% 	=
% 	-\frac{i\beta}{\hbar}
% 	\xi 
% 	\left[
% 		1
% 		+
% 		(-\xi ^2 \omega_{10} ^2)
% 		+
% 		(-\xi ^2 \omega_{10} ^2)^2
% 		+
% 		O(-\xi ^2 \omega_{10} ^2)^3
% 	\right]
% 	\\
% 	&=
% 	-\frac{i\beta}{\hbar}
% 	\left[
% 		\xi
% 		-\xi ^3 \omega_{10} ^2
% 		+
% 		\xi ^5 \omega_{10} ^4
% 		+
% 		O(\xi ^7 \omega_{10} ^6)
% 	\right]
% 	=
% 	-\frac{i\beta}{\hbar}\xi +
% 	O(\xi^3)
% 	=
% 	-\frac{i\beta}{\eta\hbar} +
% 	O(\eta^{-3})
% \end{align}

\section*{Time-dependent perturbation theory for the quantum harmonic oscillator}


\paragraph{Problem:}

Consider the quantum harmonic oscillator:
\begin{align}
	\hat H_0
	=
	\frac{1}{2m}\hat p^2
	+
	\frac{1}{2}m \omega^2 \hat x^2,
\end{align}
which is being perturbed by a linear electric field $\mathcal{E}(t)$ that is turned on and then turned back off:
\begin{align}
	\hat V(t)
	=
	q \mathcal{E}(t) \hat x,
\end{align}
where,
\begin{align}
	\mathcal{E}(t)
	=
	\frac{\mathcal{E}_0}{\tau\sqrt{\pi}}
	e^{-(t/\tau)^2},
\end{align}
where $q\mathcal{E}_0 \ll \sqrt{m\omega\hbar}$.
If the system starts in the ground state $\ket{0}$ at $t \to -\infty$, find the probability the state will be in an excited state $\ket{n}$ using first-order time-dependent perturbation theory.
Examine the behavior when $\tau \ll 1/\omega$ and $\tau \gg 1/\omega$.


\paragraph{Solution:}

As usual, we introduce ladder operators:
\begin{align}
	\hat a
	=
	\sqrt{\frac{m \omega}{2 \hbar}} ( \hat x + \frac{i}{m\omega}\hat p),
	\qquad
	\hat a^\dagger
	=
	\sqrt{\frac{m \omega}{2 \hbar}} ( \hat x - \frac{i}{m\omega}\hat p),
\end{align}
\begin{align}
	\hat x = \sqrt{
		\frac{\hbar}{2m \omega}
	}
	(\hat a^\dagger +\hat a),
	\qquad
	\hat p =i \sqrt{
		\frac{\hbar m \omega}{2}
		% \frac{\hbar}{2m \omega}
	}
	(\hat a^\dagger - \hat a),
\end{align}
\begin{align}
	\hat a \ket{n} = \sqrt{n} \ket{n-1}
	,
	\qquad
	\hat a^\dagger \ket{n} = \sqrt{n+1}\ket{n+1},
	\qquad
	\hat a \ket{0} = 0.
\end{align}
This makes the relevant matrix elements of $\hat V(t)$:
\begin{align}
	V_{mn}(t)
	&=\nonumber
	\braket{m|\hat V(t)|n}
	=
	q \mathcal{E}(t)
	\bra{m}
	\hat x
	\ket{n}
	=
	q \mathcal{E}(t)
	\sqrt{\frac{\hbar}{2m\omega}}
	\bra{m}
	(\hat a^\dagger + \hat a)
	\ket{n}
	=
	q \mathcal{E}(t)
	\sqrt{\frac{\hbar}{2m\omega}}
	\bra{m}
	(\sqrt{n+1}\ket{n+1}+ \sqrt{n}\ket{n-1})
	\\
	&
	=
	q\mathcal{E}(t)
	\sqrt{\frac{\hbar}{2m\omega}}
	\left(
	\sqrt{n+1} \delta_{m,n+1}
	+
	\sqrt{n} \delta_{m,n-1}
	\right).
\end{align}
For the ground state,
\begin{align}
	V_{n0}(t)
	=
	q\mathcal{E}(t)
	\sqrt{\frac{\hbar}{2m\omega}}
	\left(
		\sqrt{1} \delta_{n1}
		+
		\sqrt{0} \delta_{n,-1}
	\right)
	=
	q\mathcal{E}(t)
	\sqrt{\frac{\hbar}{2m\omega}}
	\delta_{n1}.
\end{align}
Note that $\omega_{0n} = \omega n$, where $\omega$ is the frequency of the oscillator.
We are left to compute the integral,
\begin{align}
	c_n^{(1)}(t \to \infty)
	&=\nonumber
	-\frac{i}{\hbar}
	\int_{-\infty}^\infty
	dt
	e^{i\omega_{n0} t}
	V_{n0}(t)
	=
	-i q
	\delta_{n1}
	\sqrt{
		\frac{1}{2m\omega\hbar}
	}
	\int_{-\infty}^\infty
	dt
	e^{in\omega t}
	\mathcal{E}(t)
	\\
	&
	=
	-i
	\delta_{n1}
	\frac{q\mathcal{E}_0}{\tau}
	\sqrt{
		\frac{1}{2\pi m\omega\hbar}
	}
	\int_{-\infty}^\infty
	dt
	e^{in\omega t - t^2/\tau^2}
	=
	-i
	\delta_{n1}
	\frac{q\mathcal{E}_0}{\tau}
	\sqrt{
		\frac{1}{2\pi m\omega\hbar}
	}
	e^{-n^2 \tau^2\omega^2/4}
	\int_{-\infty}^\infty
	dt
	e^{-(t/\tau - i n \tau \omega/2)^2}
	\\
	&
	=
	-i
	\delta_{n1}
	\frac{q\mathcal{E}_0}{\tau}
	\sqrt{
		\frac{1}{2\pi m\omega\hbar}
	}
	e^{-n^2 \tau^2\omega^2/4}
	\int_{-\infty}^\infty
	dt
	e^{-(t/\tau)^2}
	=
	-i
	\delta_{n1}
	q
	\mathcal{E}_0
	\sqrt{
		\frac{1}{2 m\omega\hbar}
	}
	e^{-n^2 \tau^2\omega^2/4}.
\end{align}
This means that the probability of transition to state $\ket{1}$ is,
\begin{align}
	p_{0 \to 1} =
	|c_1^{(1)}(t \to \infty)|^2 
	=
	\frac{q^2\mathcal{E}_0^2}{2 m \omega \hbar}
	e^{-\tau^2 \omega^2/2},
\end{align}
and the probability of transition to any other excited state is $0$.
When $\tau \to 0$ (i.e., $\tau \ll 1/\omega$), $V(t)$ becomes a delta perturbation and $c_1^{(1)}(t \to \infty)
=
-iq\mathcal{E}_0/\sqrt{2m \omega \hbar}$ which is consistent with the integral $-(i \mathcal{E}_0q/\sqrt{2m \omega \hbar}) \int_{-\infty}^\infty 
dt
e^{i\omega t}
\delta(t).
$
When $\tau \gg 1/\omega$, we see $|\mathcal{E}'(t)| = \frac{2 |t| \mathcal{E}_0 e^{-\frac{t^2}{\tau ^2}}}{\sqrt{\pi } \tau ^3}
\leq
\sqrt{\frac{2}{e \pi }}\frac{ \mathcal{E}_0}{\tau ^2}
$,
so the perturbation is varied very slowly.
In this case, $c_1^{(1)}(t \to \infty) = 0$, so there is no change made after a long time.
This can be seen as a case of the adiabatic theorem.

\section*{Appendix: How do we know what functions have a $\delta$ distribution behavior as a limit?}

The $\delta$ distribution function is defined so that $\delta(x) = 0$ for all $x \ne 0$, and for all functions $g$,
\begin{align}
	\int_{-\infty}^\infty \delta(x) g(x) dx = g(0).
\end{align}
Equivalently,
\begin{align}
	\int_{-\varepsilon}^{\varepsilon} \delta(x) g(x) dx = g(0),
\end{align}
for all $\varepsilon > 0$.
If $f_\eta$ is a collection of functions parameterized by $\eta$, $f_\eta$ converges to a $\delta$ distribution function as $\eta \to a$ if,
\begin{align}
	\lim_{\eta \to a}
	\int_{-\infty}^\infty g(x) f_\eta(x) dx
	=
	g(0),
\end{align}
for all functions $g$.
In most cases, $a = \infty$ or $0$.
Here are two nice ways to check this behavior:

\paragraph{Taylor series:}

If we play fast-and-loose mathematically, we can say that all functions $g$ of interest can be written as,
\begin{align}
	g(x)
	=
	g(0)
	+
	\sum_{n=1}^\infty g^{(n)}(0) \frac{x^n}{n!},
\end{align}
in some neighborhood of $x = 0$.
Therefore, if we can show,
\begin{gather}
	\lim_{\eta \to a}
	f_\eta(x) = 0, \qquad x \ne 0
	\\
	\lim_{\eta \to a}
	\int_{-\infty}^\infty
	f_\eta(x)
	dx
	=
	1\\
	\lim_{\eta \to a}
	\int_{-\infty}^\infty
	f_\eta(x)
	x^n
	dx
	= 0, \qquad n \geq 1
\end{gather}
then,
\begin{align}
	\lim_{\eta \to a}\int_{-\infty}^\infty g(x) f_\eta (x)
	=
	g(0)
	\lim_{\eta \to a}
	\int_{-\infty}^\infty
	f_\eta(x)
	dx
	+
	\sum_{n=1}^\infty \frac{g^{n}(0)}{n!}
	\lim_{\eta \to a} \int_{-\infty}^\infty f_\eta(x) x^n dx
	=
	g(0).
\end{align}

\paragraph{Fourier transform:}

Consider that,
\begin{align}
	\int_{-\infty}^\infty e^{-2\pi ikx} \delta(x) dx
	=
	1,
\end{align}
which says that the Fourier transform of the $\delta$ distribution function is $1$, so by the Fourier inversion theorem,
\begin{align}
	\delta(x)
	=
	\int_{-\infty}^\infty e^{2\pi ikx}dk
	=
	\frac{1}{\sqrt{2\pi}}
	\int_{-\infty}^{\infty}
	e^{ikx}
	dk.
\end{align}
Therefore, $f_\eta$ converges to a $\delta$ distribution function as $\eta \to a$ if,
\begin{align}
	\lim_{\eta \to a}
	\mathcal{F}[f_\eta](k)
	=
	\lim_{\eta \to a}
	\int_{-\infty}^\infty 
	e^{ikx} f_\eta(k)dk
	=
	\frac{1}{\sqrt{2\pi}}.
\end{align}


% If $f_\eta(x)$ has all derivatives defined at $x = 0$ for $\eta \ne a$, then we can write,
% \begin{align}
% 	f_\eta (x)
% 	=
% 	f_\eta (0)
% 	+
% 	\sum_{n = 1}^\infty f^{(n)}_\eta (0) \frac{x^n}{n!}.
% \end{align}
% If we are able to show,
% \begin{align}
% 	\lim_{\eta \to a} \int_{-\infty}^\infty
% 	f_\eta(0) g(x) dx
% \end{align}


\end{document}






