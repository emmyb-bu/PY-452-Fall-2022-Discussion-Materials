\documentclass[10pt]{article}
% \usepackage{geometry}
% \geometry{margin=0.2in}
\usepackage[utf8]{inputenc}

\nonstopmode
% \usepackage{minted}[cache=false]
\usepackage{graphicx} % Required for including pictures
\usepackage[figurename=Figure]{caption}
% \usepackage{float}    % For tables and other floats
\usepackage{amsmath}  % For math
\usepackage{amssymb}  % For more math
\usepackage{fullpage} % Set margins and place page numbers at bottom center
% \usepackage{paralist} % paragraph spacing
% \usepackage{subfig}   % For subfigures
%\usepackage{physics}  % for simplified dv, and 
% \usepackage{enumitem} % useful for itemization
% \usepackage{siunitx}  % standardization of si units
\usepackage{hyperref}
% \usepackage{mmacells}
% \usepackage{listings}
% \usepackage{svg}
% \usepackage{xcolor, soul}
\usepackage{bm}
\usepackage{braket}
\usepackage{cancel}
% \usepackage{setspace}
% \usepackage{listings}
% \usepackage{listings}
% \usepackage[autoload=true]{jlcode}
% \usepackage{pygmentize}

% \definecolor{cambridgeblue}{rgb}{0.64, 0.76, 0.68}

% \sethlcolor{cambridgeblue}

\usepackage[margin=1.8cm]{geometry}
\newcommand{\C}{\mathbb C}
\newcommand{\D}{\bm D}
\newcommand{\R}{\mathbb R}
\newcommand{\Q}{\mathbb Q}
\newcommand{\Z}{\mathbb Z}
\newcommand{\N}{\mathbb N}
\newcommand{\PP}{\mathbb P}
\newcommand{\A}{\mathbb A}
\newcommand{\F}{\mathbb F}
\newcommand{\1}{\mathbf 1}
\newcommand{\ip}[1]{\left< #1 \right>}
\newcommand{\abs}[1]{\left| #1 \right|}
\newcommand{\norm}[1]{\left\| #1 \right\|}

\def\Tr{{\rm Tr}}
\def\tr{{\rm tr}}
\def\Var{{\rm Var}}
\def\calA{{\mathcal A}}
\def\calB{{\mathcal B}}
\def\calD{{\mathcal D}}
\def\calE{{\mathcal E}}
\def\calG{{\mathcal G}}
\def\from{{:}}
\def\lspan{{\rm span}}
\def\lrank{{\rm rank}}
\def\bd{{\rm bd}}
\def\acc{{\rm acc}}
\def\cl{{\rm cl}}
\def\sint{{\rm int}}
\def\ext{{\rm ext}}
\def\lnullity{{\rm nullity}}
% \DeclareSIUnit\clight{\text{\ensuremath{c}}}
% \DeclareSIUnit\fm{\femto\m}
% \DeclareSIUnit\hplanck{\text{\ensuremath{h}}}


% \lstdefinelanguage{julia}%
%   {morekeywords={abstract,break,case,catch,const,continue,do,else,elseif,%
%       end,export,false,for,function,immutable,import,importall,if,in,%
%       macro,module,otherwise,quote,return,switch,true,try,type,typealias,%
%       using,while},%
%    sensitive=true,%
% %    alsoother={$},%
%    morecomment=[l]\#,%
%    morecomment=[n]{\#=}{=\#},%
%    morestring=[s]{"}{"},%
%    morestring=[m]{'}{'},%
% }[keywords,comments,strings]%

% \lstset{%
%     language         = Julia,
%     basicstyle       = \ttfamily,
%     keywordstyle     = \bfseries\color{blue},
%     stringstyle      = \color{magenta},
%     commentstyle     = \color{ForestGreen},
%     showstringspaces = false,
% }

% $
\begin{document}
\begin{center}
	\hrule
	\vspace{.4cm}
	{\textbf { \large CAS PY 452 --- Quantum Physics II}}
\end{center}
Emmy Blumenthal \hspace{\fill} \hspace{\fill}  \textbf{} Discussion Notes\  \\
\textbf{Date:}\  October 26, 2022   \hspace{\fill} \textbf{Email:}\ emmyb320@bu.edu

\vspace{.4cm}
\hrule





\section*{Degenerate perturbation theory: $2 \times 2$}

\paragraph{Problem:}

Find the energies, stationary states, and degeneracies of a spin-$\frac{1}{2}$ particle quadratically coupled to a magnetic field:
\begin{align}
	\hat H_0 = h_z \hat S_z^2,
\end{align}
where $\hat S_z$ is the $z$-spin operator for spin-$\frac{1}{2}$ and $h_z > 0$ is a constant with units of dimension $[\text{energy}]^{-1} \times [\text{time}]^{-2}$.
Next, perturb the Hamiltonian as $\hat H_0 \to \hat H_0 + \hat H'$ where,
\begin{align}
	\hat H' = \lambda \hat S_y,
\end{align}
where $\lambda > 0$ is a constant with units of dimension $[\text{time}]^{-1}$ and $\lambda \ll h_z \hbar$.
Find the first-order and second-order perturbed energies and the first-order perturbed states.
Finally, compute the expected value of $\hat S_x$ for each perturbed state.

\paragraph{Solution:}

In the basis $\{\ket{\uparrow},\ket{\downarrow}\}$, the spin-$\frac{1}{2}$ operators are,
\begin{gather}
	[\hat S_x]_{\{\ket{\uparrow},\ket{\downarrow}\}}
	=
	\frac{\hbar}{2} \left(
		\begin{array}{cc}
			0 & 1\\
			1 & 0
		\end{array}
	\right),
	\qquad
	[\hat S_y]_{\{\ket{\uparrow},\ket{\downarrow}\}}
	=
	\frac{\hbar}{2} \left(
		\begin{array}{cc}
			0 & -i\\
			i & 0
		\end{array}
	\right),
	\qquad
	[\hat S_z]_{\{\ket{\uparrow},\ket{\downarrow}\}}
	=
	\frac{\hbar}{2} \left(
		\begin{array}{cc}
			1 & 0\\
			0 & -1
		\end{array}
	\right),
	% \qquad
	\\
	[\hat S_z^2]_{\{\ket{\uparrow},\ket{\downarrow}\}}
	=
	\frac{\hbar^2}{4} \left(
		\begin{array}{cc}
			1 & 0\\
			0 & 1
		\end{array}
	\right),
\end{gather}
In case this matrix notation is unclear: for example, this means $\hat S_x \ket{\uparrow} = \frac{\hbar}{2} \ket{\downarrow}$ and $\hat S_y (\ket{\uparrow} + \ket{\downarrow}) = \frac{\hbar}{2} (-i \ket{\uparrow} + i \ket{\downarrow})$.
To find the energies, we find the eigenvalues of $\hat H_0$:
\begin{align}
	\det(\hat H_0 - E I)
	=
	\left|
\begin{array}{cc}
 \frac{\hbar ^2 h_z}{4}-E & 0 \\
 0 & \frac{\hbar ^2 h_z}{4}-E \\
\end{array}
\right|
=
\left(
	\frac{\hbar ^2 h_z}{4}-E
\right)^2
=
0
\implies
E = \frac{\hbar ^2 h_z}{4}.
\end{align}
To find the stationary states, we solve the eigenvalue equation:
\begin{gather}
	\hat H_0 (\alpha_\uparrow \ket{\uparrow} + \alpha_\downarrow \ket{\downarrow})
	=
	E(\alpha_\uparrow \ket{\uparrow} + \alpha_\downarrow \ket{\downarrow})
\\
	% \alpha_\uparrow \hat H_0\ket{\uparrow} + \alpha_\downarrow \hat H_0 \ket{\downarrow}
	% =
	\alpha_\uparrow \frac{h_z \hbar^2}{4} \ket{\uparrow} + \alpha_\downarrow \frac{h_z \hbar^2}{4}\ket{\downarrow}
	=
	\frac{\hbar ^2 h_z}{4}(\alpha_\uparrow \ket{\uparrow} + \alpha_\downarrow \ket{\downarrow}).
\end{gather}
We see that any choice of $\alpha_\uparrow, \alpha_\downarrow$ would satisfies this eigenvalue equation and that there are two distinct eigenvectors we can form using choices of $\alpha_\uparrow, \alpha_\downarrow$.
This means that there is a degeneracy and $E_1^{(0)} = E_2^{(0)} = \hbar^2 h_z / 4$.
To make our computations easier in the basis $\{\ket{\uparrow}, \ket{\downarrow}\}$, we will choose $\alpha_\uparrow =1, \alpha_\downarrow = 0$ and $\alpha_\uparrow = 0, \alpha_\downarrow = 1$.
Our degenerate ground states are then $\ket{\phi_1^{(0)}} = \ket{\uparrow}$ and $\ket{\phi_2^{(0)}} = \ket{\downarrow}$.
The matrix elements of the Hamiltonian in the basis $\{\ket{\phi_1^{(0)}},\ket{\phi_2^{(0)}}\}$ is,
\begin{align}
	[\hat H_0]_{\{\ket{\phi_1^{(0)}},\ket{\phi_2^{(0)}}\}}
	=
	\left(
		\begin{array}{cc}
			\braket{\phi_1^{(0)} | \hat H_0 | \phi_1^{(0)}} & \braket{\phi_1^{(0)} | \hat H_0 | \phi_2^{(0)}}\\
			\braket{\phi_2^{(0)} | \hat H_0 | \phi_1^{(0)}} & \braket{\phi_2^{(0)} | \hat H_0 | \phi_2^{(0)}}
		\end{array}
	\right)
	=
	\left(
	\begin{array}{cc}
		\frac{\hbar^2h_z}{4} & 0\\
		0 & \frac{\hbar^2h_z}{4}\\
	\end{array}
	\right)
\end{align}

In order to perturb the Hamiltonian, we will need the following matrix elements (in the basis  $\{\ket{\phi_1^{(0)}},\ket{\phi_2^{(0)}}\}$):
\begin{align}
	[\hat H']_{\{\ket{\phi_1^{(0)}},\ket{\phi_2^{(0)}}\}}
	=
	\left(
		\begin{array}{cc}
			\braket{\phi_1^{(0)} | \hat H' | \phi_1^{(0)}} & \braket{\phi_1^{(0)} | \hat H' | \phi_2^{(0)}}\\
			\braket{\phi_2^{(0)} | \hat H' | \phi_1^{(0)}} & \braket{\phi_2^{(0)} | \hat H' | \phi_2^{(0)}}
		\end{array}
	\right)
	=
	\left(
\begin{array}{cc}
 0 & -\frac{1}{2} i \lambda  \hbar  \\
 \frac{1}{2} i \lambda  \hbar & 0 \\
\end{array}
\right).
\end{align}
When calculating the first-order correction to the states, we will have to evaluate the terms $\frac{\braket{\phi_2^{(0)} | \hat H' | \phi_1^{(0)}}}{E_1^{(0)} - E_2^{(0)}} \ket{\phi_1^{(0)}}$ and $\frac{\braket{\phi_1^{(0)} | \hat H' | \phi_2^{(0)}}}{E_2^{(0)} - E_1^{(0)}} \ket{\phi_2^{(0)}}$, but because the denominators are zero, we have a problem.
Our rule-of-thumb tells us though that if we can get the numerators to be zero, we can ignore these problematic terms.
As we saw above, for $\hat H_0$, we can choose a different linear combination of eigenvectors and it will still be an eigenvector (i.e., different choice of $\alpha_\uparrow,\alpha_\downarrow$).
Therefore, if we diagonalize $\hat H'$ so that the problematic terms are zero, $\hat H_0$ will be left unchanged.
The eigenvectors of $\hat H'$ are $\ket{\phi_a^{(0)}}=\frac{1}{\sqrt{2}} \left(i\ket{\uparrow} +  \ket{\downarrow}\right)$ and $\ket{\phi_b^{(0)}}=\frac{1}{\sqrt{2}} \left(i\ket{\uparrow} -  \ket{\downarrow}\right)$, so if we change to the basis $\{\ket{\phi_a^{(0)}},\ket{\phi_b^{(0)}}\}$,
\begin{align}
	[\hat H']_{\{\ket{\phi_a^{(0)}},\ket{\phi_b^{(0)}}\}}
	&=
	\left(
		\begin{array}{cc}
			\braket{\phi_a^{(0)} | \hat H' | \phi_a^{(0)}} & \braket{\phi_a^{(0)} | \hat H' | \phi_b^{(0)}}\\
			\braket{\phi_b^{(0)} | \hat H' | \phi_a^{(0)}} & \braket{\phi_b^{(0)} | \hat H' | \phi_b^{(0)}}
		\end{array}
	\right)
	=
	\left(
\begin{array}{cc}
- \frac{\lambda\hbar}{2} & 0\\
0 & \frac{\lambda \hbar}{2}
\end{array}
\right),
\\
[\hat H_0]_{\{\ket{\phi_a^{(0)}},\ket{\phi_b^{(0)}}\}}
&=
\left(
	\begin{array}{cc}
		\braket{\phi_a^{(0)} | \hat H_0 | \phi_a^{(0)}} & \braket{\phi_a^{(0)} | \hat H_0 | \phi_b^{(0)}}\\
		\braket{\phi_b^{(0)} | \hat H_0 | \phi_a^{(0)}} & \braket{\phi_b^{(0)} | \hat H_0 | \phi_b^{(0)}}
	\end{array}
\right)
=
\left(
\begin{array}{cc}
	\frac{\hbar^2h_z}{4} & 0\\
	0 & \frac{\hbar^2h_z}{4}\\
\end{array}
\right).
\end{align}
By changing to the eigenvector basis of $\hat H'$ that is also an eigenvector basis for $\hat H_0$, we have diagonalized both $\hat H'$ and $\hat H_0$, meaning that the off-diagonal elements (i.e., the problematic elements) of $\hat H'$ are zero.
This means that $\ket{\phi_a^{(0)}},\ket{\phi_b^{(0)}}$ are the ``good states.''
We can now proceed with our calculations:
\begin{align}
	E_a^{(1)}
	&=
	\braket{\phi_{a}^{(0)} | \hat H' | \phi_{a}^{(0)}}
	=
	-\frac{\lambda\hbar}{2}
	\\
	E_b^{(1)}
	&=
	\braket{\phi_{b}^{(0)} | \hat H' | \phi_{b}^{(0)}}
	=
	+\frac{\lambda\hbar}{2}
	\\
	E_a^{(2)}
	&=
	\sum_{m \ne a}
	\frac{|\braket{\phi_{m}^{(0)} | \hat H' | \phi_{a}^{(0)}}|^2}{E_{m}^{(0)}-E_{a}^{(0)}}
	=
	\cancelto{0}{\frac{|\braket{\phi_{b}^{(0)} | \hat H' | \phi_{a}^{(0)}}|^2}{E_{b}^{(0)}-E_{a}^{(0)}}}
	=
	0
	\\
	E_b^{(2)}
	&=
	\sum_{m \ne b}
	\frac{|\braket{\phi_{m}^{(0)} | \hat H' | \phi_{b}^{(0)}}|^2}{E_{m}^{(0)}-E_{b}^{(0)}}
	=
	\cancelto{0}{\frac{|\braket{\phi_{a}^{(0)} | \hat H' | \phi_{b}^{(0)}}|^2}{E_{a}^{(0)}-E_{b}^{(0)}}}
	=
	0
	\\
	% E_a^{(2)}
	\ket{\phi^{(1)}_a}
	&=
	\sum_{m \ne a}
	\frac{\braket{\phi_{m}^{(0)} | \hat H' | \phi_{a}^{(0)}}}{E_{m}^{(0)}-E_{a}^{(0)}}
	\ket{\phi_m^{(0)}}
	=
	\cancelto{0}{\frac{\braket{\phi_{b}^{(0)} | \hat H' | \phi_{a}^{(0)}}}{E_{b}^{(0)}-E_{a}^{(0)}}} 
	\ket{\phi_b^{(0)}}
	=
	0
	\\
	\ket{\phi^{(1)}_b}
	&=
	\sum_{m \ne b}
	\frac{\braket{\phi_{m}^{(0)} | \hat H' | \phi_{b}^{(0)}}}{E_{m}^{(0)}-E_{b}^{(0)}}
	\ket{\phi_m^{(0)}}
	=
	\cancelto{0}{\frac{\braket{\phi_{a}^{(0)} | \hat H' | \phi_{b}^{(0)}}}{E_{a}^{(0)}-E_{b}^{(0)}}}
	\ket{\phi_a^{(0)}}
	=
	0
	\\
	E_a
	&=
	\frac{\hbar^2h_z}{4}
	-\frac{\lambda\hbar}{2}
	+
	0
	+
	\cdots
	\\
	E_b
	&=
	\frac{\hbar^2h_z}{4}
	+\frac{\lambda\hbar}{2}
	+
	0
	+
	\cdots
	\\
	\ket{\phi_a}
	&=
	\ket{\phi_a^{(0)}}
	+
	0
	+
	\cdots
	=
	\frac{1}{\sqrt{2}} \left(i\ket{\uparrow} +  \ket{\downarrow}\right)
	+
	0
	+
	\cdots
	\\
	\ket{\phi_b}
	&=
	\ket{\phi_b^{(0)}}
	+
	0
	+
	\cdots
	=
	\frac{1}{\sqrt{2}} \left(i\ket{\uparrow} -  \ket{\downarrow}\right)
	+
	0
	+
	\cdots
\end{align}
We see that there are no second-order corrections to the energies or first-order corrections to the states because of our diagonalization. 
However, from this process, we have obtained the resulting states that arise when the perturbation is turned on and the energy levels separate according to the first-order corrections to the energies.
Now, we can calculate,
\begin{align}
	\braket{\phi_a | \hat S_x |\phi_a}
	&=
	\left(\frac{1}{\sqrt{2}} \left(-i\bra{\uparrow} +  \bra{\downarrow}\right)\right)
	\hat S_x
	\left(
		\frac{1}{\sqrt{2}} \left(i\ket{\uparrow} +  \ket{\downarrow}\right)
	\right)
	=
	-
	\frac{\hbar}{2},
	\\
	\braket{\phi_b | \hat S_x |\phi_b}
	&=
	\left(\frac{1}{\sqrt{2}} \left(-i\bra{\uparrow} -  \bra{\downarrow}\right)\right)
	\hat S_x
	\left(
		\frac{1}{\sqrt{2}} \left(i\ket{\uparrow} -  \ket{\downarrow}\right)
	\right)
	=
	\frac{\hbar}{2},
\end{align}
so we see that despite there being no explicit correction to our state according to the computations, there are tangible results found by identifying the ``good states.''

% To find the stationary states, we solve 
% \begin{align}
% 	\hat S_z \ket{\psi}
% \end{align}


\newpage






\section*{Degenerate perturbation theory: $2 \times 2$ in a $3 \times 3$ system}

\paragraph{Problem:}

Repeat what we did in the previous problem but now we consider spin-1 and
\begin{align}
	\hat H_0 = - h_z \hat S_z^2,
	\qquad 
	\hat H' = \lambda (\hat S_x^2 +\hbar \hat S_x).
\end{align}
Be careful to note that $\lambda$ has different units than in the previous problem.

\paragraph{Solution:}

In the basis\footnote{
	These states are shorthand for the simultaneous eigenstates of $\hat S^2$ and $S_z$: $\ket{\downarrow }\equiv\ket{1,-1},\ket{0} \equiv \ket{1,0},\ket{\uparrow} \equiv\ket{1,1}$.
}
$\beta = \{\ket{\uparrow}, \ket{0}, \ket{\downarrow}\}$, the spin-1 operators are,
\begin{gather}
	[\hat S_x]_\beta
	=
	\frac{\hbar}{\sqrt{2}}
	\left(
		\begin{array}{ccc}
		 0 & 1 & 0 \\
		 1 & 0 & 1 \\
		 0 & 1 & 0 \\
		\end{array}
	\right),
	\qquad
	[\hat S_y]_\beta
	=
	\frac{\hbar}{i\sqrt{2}}
	\left(
\begin{array}{ccc}
 0 & 1 & 0 \\
 -1 & 0 & 1 \\
 0 & -1 & 0 \\
\end{array}
\right),
\qquad
[\hat S_z]_\beta
=
\hbar
\left(
\begin{array}{ccc}
 1 & 0 & 0 \\
 0 & 0 & 0 \\
 0 & 0 & -1 \\
\end{array}
\right)\\
[\hat S_z^2]_\beta
=
\hbar^2
\left(
\begin{array}{ccc}
 1 & 0 & 0 \\
 0 & 0 & 0 \\
 0 & 0 & 1 \\
\end{array}
\right).
\end{gather}
The un-perturbed and perturbing Hamiltonians in this basis are,
\begin{align}
	[\hat H_0]_\beta
	=h_z\hbar^2
	\left(
	\begin{array}{ccc}
	 -1 & 0 & 0 \\
	 0 & 0 & 0 \\
	 0 & 0 & -1 \\
	\end{array}
	\right)
	,\qquad
	[\hat H']_\beta
	=
	\lambda
	\hbar^2
	\left(
		\begin{array}{ccc}
		 \frac{1}{2} & \frac{1}{\sqrt{2}} & \frac{1}{2} \\
		 \frac{1}{\sqrt{2}} & 1 & \frac{1}{\sqrt{2}} \\
		 \frac{1}{2} & \frac{1}{\sqrt{2}} & \frac{1}{2} \\
		\end{array}
		\right).
\end{align}
We see that because $[\hat H_0]_\beta$ is already diagonalized, there are two degenerate stationary states $\ket{\phi_1^{(0)}} = \ket{\uparrow}$ and $\ket{\phi_2^{(0)}} =\ket{\downarrow}$ with a common energy $E_1^{(0)} = E_2^{(0)} = -h_z \hbar^2$ and another stationary state $\ket{\phi_3^{(0)}} = \ket{0}$ with energy $E_3^{(0)} = 0$.
We can put the un-perturbed Hamiltonian and perturbing Hamiltonian in the new basis $\gamma = \{\ket{\phi_1^{(0)}},\ket{\phi_2^{(0)}},\ket{\phi_3^{(0)}}\}$, where states with common energies are grouped together:
\begin{align}
	[\hat H_0]_\gamma
	=h_z\hbar^2
	\left(
	\begin{array}{ccc}
	 -1 & 0 & 0 \\
	 0 & -1 & 0 \\
	 0 & 0 & 0 \\
	\end{array}
	\right)
	,\qquad
	[\hat H']_\gamma
	=
	\lambda \hbar^2
	\left(
\begin{array}{ccc}
 \frac{1}{2} & \frac{1}{2} & \frac{1}{\sqrt{2}} \\
 \frac{1}{2} & \frac{1}{2} & \frac{1}{\sqrt{2}} \\
 \frac{1}{\sqrt{2}} & \frac{1}{\sqrt{2}} & 1 \\
\end{array}
\right).
\end{align}
Note the matrix representations for an operator $A$ in the bases $\beta, \gamma$ are related by $[A]_\gamma = O_{\beta \gamma}^\dagger [A]_\beta O_{\beta \gamma}$ where $O_{\beta\gamma}= \left(
	\begin{array}{ccc}
	 1 & 0 & 0 \\
	 0 & 0 & 1 \\
	 0 & 1 & 0 \\
	\end{array}
	\right)$.
The $2 \times 2$ block in the upper left of matrices $	[\hat H_0]_\gamma$ and $[\hat H']_\gamma$ is known as the `degenerate block' with energy $E_1^{(0)} = E_2^{(0)} = - h_z \hbar^2$.
In order to perform perturbation theory and not have to deal with terms that blow up, we need the terms on the off-diagonal within the degenerate block to be zero.
This is done by forming appropriate linear combinations of the basis states $\ket{\phi_1^{(0)}}$ and $\ket{\phi_2^{(0)}}$ that diagonalize the $2 \times 2$ block.
To perform the diagonalization, we find the eigenvectors of the $2 \times 2$ block: $\frac{\lambda\hbar^2}{2}\left(
	\begin{array}{cc}
		1 & 1\\
		1 & 1
	\end{array}
\right)$.
The eigenvectors of the block are $\frac{1}{\sqrt{2}} \left(\begin{array}{c}
	1\\1
\end{array}\right)$ and $\frac{1}{\sqrt{2}} \left(\begin{array}{c}
	1\\-1
\end{array}\right)$ which can be found by finding the eigenvalues then solving the eigenvector equation.
This tells us that our new basis vectors should be chosen so that $\ket{\phi_1^{(0)}}$ and $\ket{\phi_2^{(0)}}$ are replaced with $\ket{\phi_a^{(0)}} = \frac{1}{\sqrt{2}}(\ket{\phi_1^{(0)}} + \ket{\phi_2^{(0)}})$ and $\ket{\phi_b^{(0)}} =  \frac{1}{\sqrt{2}}(\ket{\phi_1^{(0)}} - \ket{\phi_2^{(0)}})$.
We call this new basis $\eta = \{\ket{\phi_a^{(0)}}, \ket{\phi_b^{(0)}}, \ket{\phi_3^{(0)}}\}$.
The change-of-basis matrix is,
\begin{align}
	O_{\gamma \eta} = 
	\left(
		\begin{array}{ccc}
		 \frac{1}{\sqrt{2}} & \frac{1}{\sqrt{2}} & 0 \\
		 \frac{1}{\sqrt{2}} & -\frac{1}{\sqrt{2}} & 0 \\
		 0 & 0 & 1 \\
		\end{array}
		\right)
\end{align}
so in the new basis the un-perturbed and perturbing Hamiltonians are,
\begin{align}
	[\hat H_0]_\eta
	&=
	O_{\gamma\eta}^\dagger
	[\hat H_0]_\gamma
	O_{\gamma\eta}
	=
	\hbar^2 h_z
	\left(
\begin{array}{ccc}
 -1 & 0 & 0 \\
 0 & -1 & 0 \\
 0 & 0 & 0 \\
\end{array}
\right)
=
\left(
	\begin{array}{ccc}
		\braket{\phi_a^{(0)} | \hat H_0 | \phi_a^{(0)}}
		&
		\braket{\phi_a^{(0)} | \hat H_0 | \phi_b^{(0)}}
		&
		\braket{\phi_a^{(0)} | \hat H_0 | \phi_3^{(0)}}
		\\
		\braket{\phi_b^{(0)} | \hat H_0 | \phi_a^{(0)}}
		&
		\braket{\phi_b^{(0)} | \hat H_0 | \phi_b^{(0)}}
		&
		\braket{\phi_b^{(0)} | \hat H_0 | \phi_3^{(0)}}
		\\
		\braket{\phi_3^{(0)} | \hat H_0 | \phi_a^{(0)}}
		&
		\braket{\phi_3^{(0)} | \hat H_0 | \phi_b^{(0)}}
		&
		\braket{\phi_3^{(0)} | \hat H_0 | \phi_3^{(0)}}
		\\
	\end{array}
\right)
,
\\
% \qquad
[\hat H']_\eta
&=
O_{\gamma\eta}^\dagger
[\hat H']_\gamma
O_{\gamma\eta}
=
\lambda\hbar^2
\left(
\begin{array}{ccc}
 1 & 0 & 1 \\
 0 & 0 & 0 \\
 1 & 0 & 1 \\
\end{array}
\right)
=
\left(
	\begin{array}{ccc}
		\braket{\phi_a^{(0)} | \hat H' | \phi_a^{(0)}}
		&
		\braket{\phi_a^{(0)} | \hat H' | \phi_b^{(0)}}
		&
		\braket{\phi_a^{(0)} | \hat H' | \phi_3^{(0)}}
		\\
		\braket{\phi_b^{(0)} | \hat H' | \phi_a^{(0)}}
		&
		\braket{\phi_b^{(0)} | \hat H' | \phi_b^{(0)}}
		&
		\braket{\phi_b^{(0)} | \hat H' | \phi_3^{(0)}}
		\\
		\braket{\phi_3^{(0)} | \hat H' | \phi_a^{(0)}}
		&
		\braket{\phi_3^{(0)} | \hat H' | \phi_b^{(0)}}
		&
		\braket{\phi_3^{(0)} | \hat H' | \phi_3^{(0)}}
		\\
	\end{array}
\right).
\end{align}
We see that by changing to the new basis $\eta$, we have made the off-diagonal elements of the $2 \times 2$ degenerate block all zero so that we may proceed with the perturbation theory calculation.
I have written out what each matrix entry represents as a matrix elements on the RHS of the above equations in order to make what we're doing and how it relates to the computations a bit more explicit.
Now, we can proceed with the perturbation theory calculation:
\begin{align}
	E_a^{(1)}
	&=
	\braket{\phi_a^{(0)} | \hat H' | \phi_a^{(0)}}
	=
	\lambda\hbar^2
	\\
	E_b^{(1)}
	&=
	\braket{\phi_b^{(0)} | \hat H' | \phi_b^{(0)}}
	=
	0
	\\
	E_3^{(1)}
	&=
	\braket{\phi_3^{(0)} | \hat H' | \phi_3^{(0)}}
	=
	\lambda\hbar^2
	\\
	E_a^{(2)}
	&=
	\sum_{m \ne a}
	\frac{|\braket{\phi_m^{(0)} | \hat H' | \phi_a^{(0)}}|^2}{E_a^{(0)} - E_m^{(0)}}
	=
	\cancelto{0}{\frac{|\braket{\phi_b^{(0)} | \hat H' | \phi_a^{(0)}}|^2}{E_a^{(0)} - E_b^{(0)}}}
	+
	\frac{\lambda\hbar^2}{-\hbar^2 h_z}
	=
	-\frac{\lambda^2\hbar^2}{h_z}
	\\
	E_b^{(2)}
	&=
	\sum_{m \ne b}
	\frac{\braket{|\phi_m^{(0)} | \hat H' | \phi_b^{(0)}}|^2}{E_b^{(0)} - E_m^{(0)}}
	=
	\cancelto{0}{\frac{|\braket{\phi_a^{(0)} | \hat H' | \phi_b^{(0)}}|^2}{E_b^{(0)} - E_a^{(0)}}}
	+
	0
	% \frac{\lambda^2\hbar^4}{-\hbar^2 h_z}
	=
	% -\frac{\lambda^2 \hbar^2}{h_z}
	0
	\\
	E_3^{(2)}
	&=
	\sum_{m \ne 3}
	\frac{|\braket{\phi_m^{(0)} | \hat H' | \phi_3^{(0)}}|^2}{E_3^{(0)} - E_m^{(0)}}
	=
	\frac{\lambda^2\hbar^4}{\hbar^2 h_z}
	+
	0
	=
	+
	\frac{\lambda^2\hbar^2}{h_z}
	\\
	\ket{\phi^{(1)}_a}
	&=
	\sum_{m \ne a}
	\frac{\braket{\phi_m^{(0)} | \hat H' | \phi_a^{(0)}}}{E_a^{(0)} - E_m^{(0)}}
	\ket{\phi_m^{(0)}}
	=
	\cancelto{0}{\frac{\braket{\phi_b^{(0)} | \hat H' | \phi_a^{(0)}}}{E_a^{(0)} - E_b^{(0)}}}
	\ket{\phi_b^{(0)}}
	-
	\frac{\lambda\hbar^2}{\hbar^2 h_z}
	\ket{\phi_3^{(0)}}
	=
	-
	\frac{\lambda}{ h_z}
	\ket{0}
	% \frac{1}{\sqrt{2}}
	% \left(
	% 	\ket{\uparrow}
	% 	+
	% 	\ket{\downarrow}
	% \right)
	\\
	\ket{\phi^{(1)}_b}
	&=
	\sum_{m \ne b}
	\frac{\braket{\phi_m^{(0)} | \hat H' | \phi_b^{(0)}}}{E_b^{(0)} - E_m^{(0)}}
	\ket{\phi_m^{(0)}}
	=
	\cancelto{0}{\frac{\braket{\phi_a^{(0)} | \hat H' | \phi_b^{(0)}}}{E_b^{(0)} - E_a^{(0)}}}
	\ket{\phi_a^{(0)}}
	+
	% \frac{\lambda\hbar^2}{\hbar^2 h_z}
	% \ket{\phi_3^{(0)}}
	0
	=
	% \frac{\lambda}{ h_z}
	% \ket{0}
	0
	\\
	\ket{\phi^{(1)}_3}
	&=
	\sum_{m \ne 3}
	\frac{\braket{\phi_m^{(0)} | \hat H' | \phi_3^{(0)}}}{E_3^{(0)} - E_m^{(0)}}
	\ket{\phi_m^{(0)}}
	=
	\frac{\lambda\hbar^2}{\hbar^2 h_z}
	\ket{\phi_a^{(0)}}
	+
	0
	=
	\frac{\lambda}{h_z} \frac{1}{\sqrt{2}}
	(\ket{\uparrow} + \ket{\downarrow}).
	% \frac{\lambda\hbar^2}{-\hbar^2 h_z}
	% \ket{\phi_b^{(0)}}
	% +
	% % \frac{\lambda\hbar^2}{-\hbar^2 h_z}
	% % \ket{\phi_a^{(0)}}
	% % % =
	% -
	% \frac{\sqrt{2}\lambda}{h_z}
	% \ket{\uparrow }
	% \cancelto{0}{\frac{\braket{\phi_a^{(0)} | \hat H' | \phi_3^{(0)}}}{E_3^{(0)} - E_a^{(0)}}}
	% \ket{\phi_a^{(0)}}
	% +
	% \frac{\lambda\hbar^2}{\hbar^2 h_z}
	% \ket{\phi_3^{(0)}}
	% =
	% \frac{\lambda}{ h_z}
	% \ket{0}
	% \frac{1}{\sqrt{2}}
	% \left(
	% 	\ket{\uparrow}
	% 	+
	% 	\ket{\downarrow}
	% \right)
	% \ket{\phi_3^{(0)}}
\end{align}
Putting this all together,
\begin{align}
	\ket{\phi_a}
	&=
	\ket{\phi_a^{(0)}}
	+
	\ket{\phi_a^{(1)}}
	+
	\cdots
	=
	\frac{1}{\sqrt{2}} (\ket{\uparrow} + \ket{\downarrow})
	-
	\frac{\lambda}{h_z} \ket{0}
	+
	\cdots
	\xrightarrow{\text{normalize}}
	\ket{\phi_a}
	=
	\frac{1}{\sqrt{1+\lambda^2/h_z^2}}
	(	\frac{1}{\sqrt{2}} (\ket{\uparrow} + \ket{\downarrow})
	-
	\frac{\lambda}{h_z} \ket{0})
	\\
	\ket{\phi_b}
	&=
	\ket{\phi_b^{(0)}}
	+
	\ket{\phi_b^{(1)}}
	+
	\cdots
	=
	\frac{1}{\sqrt{2}} (\ket{\uparrow} - \ket{\downarrow})
	+
	0
	% \frac{\lambda}{h_z} \ket{0}
	+
	\cdots
	\xrightarrow{\text{already normalized}}
	\ket{\phi_b}
	=	\frac{1}{\sqrt{2}} (\ket{\uparrow} - \ket{\downarrow})
	\\
	\ket{\phi_3}
	&=
	\ket{\phi_3^{(0)}}
	+
	\ket{\phi_3^{(1)}}
	+
	\cdots
	=
	\ket{0}
	+
	\frac{\lambda}{h_z\sqrt{2}}
	(\ket{\uparrow}+\ket{\downarrow})
	+
	\cdots
	\xrightarrow{\text{normalize}}
	\ket{\phi_3}
	=
	\frac{1}{\sqrt{1+\lambda^2/h_z^2}}
	(
		\frac{\lambda}{h_z\sqrt{2}}
		(\ket{\uparrow}+\ket{\downarrow})
		+
		\ket{0}
	)
	\\
	E_a
	&=
	E_a^{(0)}
	+
	E_a^{(1)}
	+
	E_a^{(2)}
	+
	=
	-\hbar^2 h_z 
	+
	\lambda\hbar^2
	-
	 \frac{\lambda^2\hbar^2}{h_z}
	+
	O((\lambda/h_z)^3)
	\\
	E_b
	&=
	E_b^{(0)}
	+
	E_b^{(1)}
	+
	E_b^{(2)}
	+
	\cdots
	=
	-\hbar^2 h_z 
	+
	0
	+
	0
	+
	\cdots
	\\
	E_3
	&=
	E_3^{(0)}
	+
	E_3^{(1)}
	+
	E_3^{(2)}
	+
	\cdots
	=
	0
	+
	\lambda \hbar^2
	+
	\lambda^2\frac{\hbar^2}{h_z}
	+
	O((\lambda/h_z)^3)
	% \cdots
\end{align}
To first order, the expected $x$-magnetizations are:
\begin{align}
	\braket{\phi_a|\hat S_x | \phi_a}
	&=
	\frac{1}{1+\lambda^2/h_z^2}
	\left(
\begin{array}{c}
 \frac{1}{\sqrt{2}} \\
 -\frac{\lambda }{h_z} \\
 \frac{1}{\sqrt{2}} \\
\end{array}
\right)^\dagger
\frac{\hbar}{\sqrt{2}}
\left(
\begin{array}{ccc}
 0 & 1 & 0 \\
 1 & 0 & 1 \\
 0 & 1 & 0 \\
\end{array}
\right)
		\left(
\begin{array}{c}
 \frac{1}{\sqrt{2}} \\
 -\frac{\lambda }{h_z} \\
 \frac{1}{\sqrt{2}} \\
\end{array}
\right)
=
-
\frac
{2 \hbar (\lambda/h_z)}
{1+(\lambda/h_z)^2}
\\
\braket{\phi_b|\hat S_x | \phi_b}
&=
\left(
\begin{array}{c}
\frac{1}{\sqrt{2}} \\
0\\
-\frac{1}{\sqrt{2}} \\
\end{array}
\right)^\dagger
\frac{\hbar}{\sqrt{2}}
\left(
\begin{array}{ccc}
0 & 1 & 0 \\
1 & 0 & 1 \\
0 & 1 & 0 \\
\end{array}
\right)
	\left(
\begin{array}{c}
\frac{1}{\sqrt{2}} \\
0 \\
-\frac{1}{\sqrt{2}} \\
\end{array}
\right)
=
0
\\
\braket{\phi_3 | \hat S_x | \phi_3}
&=
\frac{1}{1+\lambda^2/h_z^2}
\left(
\begin{array}{c}
 \frac{\lambda }{\sqrt{2} h_z} \\
 1 \\
 \frac{\lambda }{\sqrt{2} h_z} \\
\end{array}
\right)^\dagger
\frac{\hbar}{\sqrt{2}}
\left(
\begin{array}{ccc}
0 & 1 & 0 \\
1 & 0 & 1 \\
0 & 1 & 0 \\
\end{array}
\right)
\left(
\begin{array}{c}
 \frac{\lambda }{\sqrt{2} h_z} \\
 1 \\
 \frac{\lambda }{\sqrt{2} h_z} \\
\end{array}
\right)
=
\frac
{2 \hbar (\lambda/h_z)}
{1+(\lambda/h_z)^2}
\end{align}
% More to come later today\dots Along with better writing/grammar, etc., \dots
% We find these linear combinations by diagonalization through finding the eigenvectors of the $2 \times 2$ block.


\newpage


\section*{Variational method and the radial equation}

\paragraph{Problem:}

Consider the 3D system of a particle in a isotropic, central, quadratic potential $V(r) =\alpha r^2$ so that its Hamiltonian is,
\begin{align}
	\hat H = -\frac{\hbar^2}{2m}\nabla^2 + \alpha r^2.
\end{align}
Provide an ansatz, apply the variational method, and find an upper bound on the ground state energy.

\paragraph{Reminder:}

When we solve the time-independent Schr\"odinger equation with a spherically-symmetric potential, we use separation of variables, write $\psi(r,\theta,\phi) = Y_\ell^m(\theta,\phi) R(r)$, and arrive at the equation,
\begin{align}
	\left[-\frac{\hbar^2}{2m} \frac{d^2}{dr^2} + \frac{\hbar^2 \ell (\ell + 1)}{2 m r^2} + V(r)\right]u(r)
	=
	Eu(r),
\end{align}
where $u(r) = r R(r)$.
This is the `reduced radial equation' and can be found in Griffiths in the section titled {\em Radial equation}.
This means that,
\begin{align}
	\hat H [Y_\ell^m (\theta,\phi)u(r) r^{-1}]
	=
	\left[-\frac{\hbar^2}{2m} \frac{d^2u}{dr^2} + \frac{\hbar^2 \ell (\ell + 1)}{2 m r^2}u(r) + V(r)u(r)\right]
	Y_\ell^m(\theta,\phi) r^{-1}.
	\label{rreqaseigenstate}
\end{align}
Also, remember the following about the spherical harmonics:
\begin{gather}
	\int_0^\pi \int_0^{2\pi} Y_{\ell_2}^{m_2}(\theta,\phi)^\ast Y_{\ell_1}^{m_1}(\theta,\phi) d\theta d\phi
	=
	\delta_{\ell_1 \ell_2}
	\delta_{m_1 m_2}
	\\
	Y_0^0 (\theta,\phi)
	=
	\frac{1}{2\sqrt{\pi}}.
\end{gather}
Therefore, if the wave-function is of the form $\psi(r,\theta,\phi) = Y_\ell^m u(r)r^{-1}$, the normalization condition becomes,
\begin{align}
	1 = \int_0^\infty\int_0^{2\pi}\int_0^\pi  \psi^\ast \psi r^2 \sin\theta d\theta d\phi dr
	=
	\left(
		\int_0^\pi \int_0^{2\pi} |Y_{\ell}^m (\theta,\phi)|^2\sin \theta d\theta d\phi
	\right)
	\left(
		\int_0^\infty [r^{-1} u(r)]^\ast r^{-1} u(r) r^2 dr
	\right)
	=
	\int_0^\infty |u(r)|^2dr.
\end{align}

\paragraph{Solution:}

Using the radial equation, we see that if we choose ansatzes of the form $\psi(r,\theta,\phi) = Y_\ell^m (\theta,\phi) u(r) r^{-1}$, computations become significantly easier.
Therefore, as an example, we will choose $u_a(r) = Ae^{-ar^2}$ where $A$ is a normalization constant and $a$ is a variational parameter.
Additionally, we will find a state with zero angular momentum: $\ell = 0$.
This makes the `whole' ansatz $\psi_a (r,\theta,\phi)= A Y_0^0(\theta,\phi) e^{-ar^2} r^{-1}
=
A r^{-1}e^{-ar^2}/(2 \sqrt{\pi})
$.
Going through our usual variational method procedure, first we normalize:
\begin{align}
	1 = \iiint |\psi_a|^2 dV
	=
	\int_0^\infty |u_a(r)|^2 dr
	=
	A^2 \int_0^\infty e^{-2ar^2} dr
	=
	A^2 \frac{1}{2} \sqrt{\frac{\pi}{2a}}
	% \frac{A^2}{2a}
	\implies
	A
	=
	\left(\frac{8a}{\pi}\right)^{1/4},
\end{align}
where we have used the formula $\int_{-\infty}^\infty e^{-b x^2} dx = \sqrt{\pi/b}$.
Next we compute the Hamiltonian acting on the ansatz according to equation \ref{rreqaseigenstate},
\begin{align}
	\hat H [\psi_a]
	\nonumber
	&=
	A Y_0^0(\theta,\phi)
	r^{-1}
	\left[
		-\frac{\hbar^2}{2 m}
		\frac{d^2}{dr^2}
		e^{-ar^2}
		+
		\frac{\hbar^2 0 (0+1)}{2 m r^2}
		e^{-ar^2}
		+
		\alpha r^2 e^{-ar^2}
	\right]\\
	&
	=
	A Y_0^0(\theta,\phi)
	r^{-1}
	\left[
		-\frac{\hbar^2}{2 m}
		\left[
			4 a^2 r^2 e^{-a r^2}-2 a e^{-a r^2}
		\right]
		+
		\frac{\hbar^2 0 (0+1)}{2 m r^2}
		e^{-ar^2}
		+
		\alpha r^2 e^{-ar^2}
	\right]
	\nonumber
	\\
	&=
	A Y_0^0 (\theta,\phi)r^{-1}
	\left[
	\left(
		\alpha-\frac{2\hbar^2a^2}{m} 
	\right)r^2 e^{-ar^2}
	+
	\frac{\hbar^2 a}{m}
	e^{-ar^2}
	\right].
\end{align}
Next, we compute the expected energy,
\begin{align}
	\braket{\psi_a | \hat H | \psi_a}
	&=\nonumber
	\int_0^{2\pi}
	\int_0^{\pi}
	\int_0^\infty 
	\psi_a^\ast \hat H[\psi_a]
	r^2 \sin\theta dr
	d\theta
	d\phi
	\\
	&=
	A^2
	\cancelto{1}{\iint|Y_0^0 (\theta,\phi)|^2d\Omega}\;
	\int_0^\infty r^{-1}e^{-a r^2}r^{-1}
	\left[
	\left(
		\alpha-\frac{2\hbar^2a^2}{m} 
	\right)r^2 e^{-ar^2}
	+
	\frac{\hbar^2 a}{m}
	e^{-ar^2}
	\right]
	r^2
	dr
	\nonumber
	\\
	&=
	A^2 
	\left[
	\left(
		\alpha-\frac{2\hbar^2a^2}{m} 
	\right)
	\int_0^\infty 
	r^2 e^{-2ar^2}
	dr
	+
	\frac{\hbar^2 a}{m}
	\int_0^\infty 
	e^{-2ar^2}
	dr
	\right].
	\nonumber
\end{align}
To calculate these integrals we use integration by parts,
\begin{gather}
	% \int_0^\infty r^2 e^{-2ar^2} dr
	% =
	\frac{d}{dr} r e^{-2ar^2}
	=
	e^{-2ar^2}
	-
	4a r^2 e^{-2ar^2}
	\implies
	\left. r e^{-2ar^2}\right|_{r=0}^{r \to \infty}
	=
	\int_0^\infty e^{-2ar^2} dr
	-
	4a 
	\int_0^\infty r^2 e^{-2ar^2} dr
	\nonumber
	\\
	0
	=
	\frac{1}{2}\sqrt{\frac{\pi}{2a}}
	-
	4a \int_0^\infty r^2 e^{-2ar^2} dr
	\implies
	\int_0^\infty r^2 e^{-2ar^2}dr
	=
	\frac{1}{8}
	\sqrt{\frac{\pi}{2a^3}}.
\end{gather}
Therefore,
\begin{align}
	\braket{\psi_a | \hat H | \psi_a}
	&=
	A^2 
	\left[
	\left(
		\alpha-\frac{2\hbar^2a^2}{m} 
	\right)
	\frac{1}{8} \sqrt{\frac{\pi}{2a^3}}
	+
	\frac{\hbar^2 a}{m}
	\frac{1}{8}\sqrt{\frac{\pi}{2a^3}}
	\right]
	=
	A^2 \sqrt{\frac{\pi}{2}}
	\left(
		\frac{\alpha }{8 a^{3/2}}+\frac{\sqrt{a} \hbar ^2}{4 m}
	\right)
	% \\
	% &
	=
	\frac{\alpha }{4 a}+\frac{a \hbar ^2}{2 m}.
	\label{eEn}
\end{align}
Next, we minimize this by computing stationary points of $\braket{\psi_a | \hat H | \psi_a}$ with respect to $a$:
\begin{align}
	\frac{\partial}{\partial a}\braket{\psi_a | \hat H | \psi_a}
	=
	\frac{\hbar ^2}{2 m}-\frac{\alpha }{4 a^2}
	=
	0
	\implies
	\frac{a^2 \hbar ^2}{2 m}-\frac{\alpha }{4}
	=
	0
	\implies
	a 
	=
	\pm
	\sqrt{\frac{m\alpha}{2\hbar^2}}.
\end{align}
We must have $a^\star = \sqrt{m\alpha/(2\hbar^2)}$ because if $a^\star$ were negative, the ansatz would not be normalizable.
Plugging this back into the average energy (line \ref{eEn}), the variational energy is,
\begin{align}
	\braket{\psi_a | \hat H | \psi_a}
	=\sqrt{\frac{\alpha\hbar^2}{2m}}.
\end{align}





% Next we compute the Hamiltonian acting on the ansatz by according to equation \ref{rreqaseigenstate},
% \begin{align}
% 	\hat H[\psi_a]
% 	&=\nonumber
% 	Y_0^0(\theta,\phi) r^{-1}
% 	\left(-\frac{\hbar^2}{2m} \frac{d^2u}{dr^2} + \frac{\hbar^2 0 (0 + 1)}{2 m r^2}u(r) + \alpha r^2 u(r)\right)
% 	\\
% 	&=\nonumber
% 	A
% 	Y_0^0(\theta,\phi) r^{-1}
% 	\left(-\frac{\hbar^2}{2m} 
% 	\left(
% 		4 a^2 r^2 e^{-a r^2}-2 a e^{-a r^2}
% 	\right) + 0 + \alpha r^2 e^{-a r^2}\right)
% 	\\
% 	&=\nonumber
% 	A
% 	Y_0^0(\theta,\phi) r^{-1}
% 	\left[
% 		\frac{a \hbar^2}{m} e^{-ar^2}
% 		+
% 		\left(
% 			\alpha
% 			-
% 			\frac{2a^2\hbar^2}{m}
% 		\right)
% 		r^2 e^{-ar^2}
% 	\right].
	% \left(
	% 	-\frac{\hbar^2}{2m}
	% 	(4 a^2 r^2 e^{-a r^2}-2 a e^{-a r^2})
	% 	+
	% 	0
	% 	+
	% 	\alpha r e^{-a r^2}
	% \right)\\
	% &=
	% A Y_0^0(\theta,\phi)r^{-1}
	% \left(
	% 	-\frac{\hbar^2}{2m}(4a^2 r^2 - 2 a)
	% 	+
	% 	\alpha r
	% \right)
	% e^{-a r^2}.
% \end{align}
% Next, we compute the expected energy,
% \begin{align}
% 	\braket{\psi_a|\hat H |\psi_a}
% 	&=\nonumber
% 	\int_0^{2\pi} \int_0^\pi \int_0^\infty 
% 	\psi_a^\ast \hat H[\psi_a] 
% 	r^2 dr d\theta d\phi
% 	\\
% 	&=\nonumber
% 	A^2 \int_0^{2\pi} \int_0^\pi 
% 	|Y_0^0(\theta,\phi)|^2 d \theta d\phi
% 	\int_0^\infty 
% 	r^{-1}
% 	e^{-ar^2}
% 	r^{-1}
% 	\left[
% 		\frac{a \hbar^2}{m}
% 		+
% 		\left(
% 			\alpha
% 			-
% 			\frac{2a^2\hbar^2}{m}
% 		\right)
% 		r^2 
% 	\right]
% 	e^{-ar^2} r^2 dr
% 	\\
% 	&=
% 	A^2 
% 	\left[
% 		\frac{a\hbar^2}{m}\int_0^\infty e^{-2ar^2} dr
% 		+
% 		\left(
% 			\alpha - \frac{2a\hbar^2}{m}
% 		\right)
% 		\int_0^\infty e^{-2ar^2} r^2dr
% 	\right].
% \end{align}
% Observe that,
% \begin{gather}
% 	\frac{d}{dr} r e^{-2ar^2}
% 	=
% 	e^{-2ar^2}
% 	-4a r^2 e^{-2ar^2}
% 	\implies
% 	\left.r e^{-2ar^2} \right|_{r=0}^{r\to\infty}
% 	=	
% 	\int_0^\infty e^{-2ar^2} dr
% 	-
% 	4a \int_0^\infty e^{-2ar^2} r^2 dr
% \nonumber\\
% \implies
% 0
% =
% \sqrt{\frac{\pi}{8 a}}
% -
% 4a \int_0^\infty e^{-2ar^2} r^2 dr
% \implies
% \int_0^\infty e^{-2ar^2} r^2 dr
% =
% \frac{1}{8}\sqrt{\frac{\pi}{2a^3}},
% \end{gather}
% so,
% \begin{align}
% 	\braket{\psi_a|\hat H |\psi_a}
% 	=
% 	\sqrt{\frac{8a^3}{\pi}}
% 	\left[
% 		\frac{a\hbar^2}{m} \sqrt{\frac{\pi}{8a}}
% 		+
% 		\left(
% 			\alpha - \frac{2a\hbar^2}{m}
% 		\right)
% 		\frac{1}{8} \sqrt{\frac{\pi}{2a^3}}
% 	\right]
% 	=
% 	\frac{\alpha}{4}
% 	+
% 	\frac{\hbar^2}{2m}a^2.
% \end{align}
% Finally, we have to minimize the expected energy,
% \begin{align}
% 	\frac{\partial}{\partial a}
% 	\braket{\psi_a | \hat H| \psi_a}
% 	=
% \end{align}
% Next, we compute the expected energy:
% \begin{align}
% 	\braket{\psi_a | \hat H | \psi_a}
% 	&=\nonumber
% 	A^2
% 	\int_0^{2\pi}
% 	\int_0^\pi
% 	\int_0^\infty
% 	Y_0^0(\theta,\phi)^\ast r^{-1} 
% 	e^{-ar^2}
% 	Y_0^0(\theta,\phi)r^{-1}
% 	\left(
% 		-\frac{\hbar^2}{2m}(4a^2 r^2 - 2 a)
% 		+
% 		\alpha r
% 	\right)
% 	e^{-a r^2}
% 	r^2 dr d\theta d\phi
% 	\\
% 	&=\nonumber
% 	A^2 
% 	\int_0^{2\pi}
% 	\int_0^\pi
% 	|Y_0^0(\theta,\phi)|^2
% 	d\theta d\phi
% 	\int_0^\infty 
% 	\left(-\frac{\hbar^2}{2m} ( 4a^2 r^2 - 2a) + \alpha r\right) e^{-2ar^2}
% 	dr
% 	\\
% 	&=%\nonumber
% 	A^2
% 	\left[
% 		-\frac{2a^2\hbar^2}{m}
% 	\int_0^\infty 
% 	e^{-2ar^2} r^2 dr
% 	+
% 	\frac{\hbar^2a}{m}
% 	\int_0^\infty 
% 	e^{-2ar^2}
% 	dr
% 	+
% 	\alpha
% 	\int_0^\infty  e^{-2\alpha r^2} r
% 	dr
% 	\right].
% 	% \int_0^\infty 
% 	% \left(-\frac{\hbar^2}{2m} ( 4a^2 r^2 - 2a) + \alpha r\right) e^{-2ar^2}
% 	% dr
% \end{align}
% Observe that,
% \begin{gather}
% 	\int_0^\infty e^{-2ar^2} r dr
% 	=
% 	\frac{1}{4a}
% 	\int_0^\infty e^{-u} du
% 	=
% 	\frac{1}{4a}
% 	\\
% 	\frac{d}{dr} e^{-2ar^2} r
% 	=
% 	e^{-2ar^2}
% 	-
% 	2 a r^2 e^{-ar^2}
% 	\implies
% 	\left.
% 	r e^{-2ar^2}	
% 	\right|_{r=0}^{r\to\infty}
% 	=
% 	\int_0^\infty e^{-2ar^2} dr
% 	-4a \int_0^\infty r^2 e^{-2ar^2}dr
% 	\\
% 	\nonumber
% 	\implies
% 	0
% 	=
% 	\frac{1}{2}\sqrt{\frac{\pi}{2a}}
% 	-
% 	4a
% 	\int_0^\infty r^2 e^{-2ar^2}dr
% 	\implies
% 	\int_0^\infty r^2 e^{-2ar^2}dr
% 	=
% 	\frac{1}{8}\sqrt{\frac{\pi}{2 a^{3}}}.
% 	% \sqrt{\frac{\pi}{8a^3}}
% \end{gather}
% Therefore, 
% \begin{align}
% 	\braket{\psi_a | \hat H | \psi_a}
% 	&=
% 	A^2
% 	\left[
% 		-\frac{2a^2\hbar^2}{m}
% 		\frac{1}{8} 
% 		\sqrt{\frac{\pi}{2a^3}}
% 		+
% 		\frac{\hbar^2 a}{m} 
% 		\sqrt{\frac{\pi}{8a}}
% 		+
% 	\frac{\alpha}{4a}
% 	\right]
% \end{align}

% 	\sqrt{2a} Y_0^0(\theta,\phi) r^{-1}
% 	\left(-\frac{\hbar^2}{2m} (-a)^2e^{-ar} + 0 + \alpha r e^{-ar}\right)
% 	\\
% 	&=
% 	\sqrt{2a} Y_0^0(\theta,\phi) r^{-1}
% 	\left(-\frac{\hbar^2a^2}{2m} + \alpha r \right)e^{-ar}.
% \end{align}
% Now, we compute the expected energy:
% \begin{align}
% 	\braket{\psi_a | \hat H | \psi_a}
% 	&\nonumber
% 	=
% 	\iiint \psi_a^\ast \hat H[\psi_a] dV
% 	=
% 	2 a \int_0^{2\pi} \int_0^\pi \int_0^\infty Y_0^0(\theta,\phi)^\ast  r^{-1} e^{-ar} Y_0^0(\theta,\phi)  r^{-1}\left(-\frac{\hbar^2a^2}{2m} + \alpha r \right)e^{-ar} r^2 dr  d\theta d\phi 
% 	\\
% 	&
% 	=
% 	2a 
% 	\left(
% 		\int_0^{2\pi} \int_0^\pi
% 		Y_0^0(\theta,\phi)^\ast Y_0^0(\theta,\phi)
% 		d\theta d\phi
% 	\right)
% 	\left(
% 		\int_0^\infty \left(-\frac{\hbar^2a^2}{2m} + \alpha r \right) e^{-2ar} dr
% 	\right)
% 	\nonumber
% 	\\
% 	&=
% 	2a 
% 	\left(
% 		-\frac{\hbar^2 a^2}{2m}\int_0^\infty e^{-2ar} dr
% 		+
% 		\alpha \int_0^\infty r e^{-2ar} dr
% 	\right).
% Observe that:
% \begin{align}
% 	\nonumber
% 	\frac{d}{dr} r e^{-2ar}
% 	=
% 	e^{-2ar}
% 	-2a r e^{-2ar}
% 	&\implies
% 	\left.r e^{-2ar}\right|_{r=0}^{r\to\infty}
% 	=
% 	\int_0^\infty 
% 	e^{-2ar}
% 	dr
% 	-
% 	2a
% 	\int_0^\infty r e^{-2ar} dr
% 	\\
% 	\implies
% 	0
% 	=
% 	\frac{1}{2a}
% 	-
% 	2a \int_0^\infty r e^{-2ar}dr
% 	&\implies
% 	\int_0^\infty r e^{-2ar} dr
% 	=
% 	\frac{1}{4a^2},
% \end{align}
% so,
% \begin{align}
% 	\braket{\psi_a | \hat H | \psi_a}
% 	=
% 	2a 
% 	\left(
% 		-\frac{\hbar^2 a^2}{2m}\frac{1}{2a}
% 		+
% 		\alpha 
% 		\frac{1}{4a^2}
% 	\right)
% 	=
% 	-
% 	\frac{\hbar^2a^2}{2m}
% 	+
% 	\frac{\alpha}{2a}.
% \end{align}
% To minimize this expected energy, we find the stationary points by taking the partial derivative with respect to $a$:
% \begin{align}
% 	\frac{\partial}{\partial a}
% 	\braket{\psi_a | \hat H | \psi_a}
% 	=
% 	-
% 	\frac{\hbar^2 a}{m}
% 	-
% 	\frac{\alpha}{2a^2}
% 	=
% 	0
% 	\implies
% 	% \frac{2\hbar^2 a^3}{m}
% 	a^3	
% 	=-\frac{m\alpha}{2 \hbar^2}.
% \end{align}
% The cubic equation $a^3 = -b$ for some $b>0$ has three solutions: $a = - b^{1/3}$, $a = (-1)^{1/3} b^{1/3}$, and $a = (-1)^{5/3} b^{1/3}$.
% For our ansatz to obey the boundary conditions, we must have $e^{-ar} \to 0$ as $r \to \infty$, meaning $\Re[a] > 0$.
% This immediately rules out $a = -b^{1/3}$.
% If we write $-1 = e^{i \pi}$, then $(-1)^{2/3} = e^{i2\pi/3} = \cos(2 \pi /3) + i \sin(2\pi/3)$ and $(-1)^{5/3} = \cos(5 \pi /3) + i \sin(5\pi/3)$.
% Because $0< 2\pi/3< \pi$, $\cos(2 \pi /3) > 0$ and because $\pi < 5\pi/3< 2 \pi$, $\cos(5\pi/3) < 0$.
% Therefore, the solution with a positive real part is $b = (-1)^{5/3} b^{1/3}$.
% To be more precise, we could use $\sin(5\pi/3) = \sqrt{3}/2$ and $\cos(5)




\end{document}






