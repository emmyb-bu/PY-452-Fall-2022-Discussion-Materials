\documentclass[10pt]{article}
% \usepackage{geometry}
% \geometry{margin=0.2in}
\usepackage[utf8]{inputenc}

\nonstopmode
% \usepackage{minted}[cache=false]
\usepackage{graphicx} % Required for including pictures
\usepackage[figurename=Figure]{caption}
\usepackage{float}    % For tables and other floats
\usepackage{amsmath}  % For math
\usepackage{amssymb}  % For more math
\usepackage{fullpage} % Set margins and place page numbers at bottom center
\usepackage{paralist} % paragraph spacing
\usepackage{subfig}   % For subfigures
%\usepackage{physics}  % for simplified dv, and 
\usepackage{enumitem} % useful for itemization
\usepackage{siunitx}  % standardization of si units
\usepackage{hyperref}
% \usepackage{mmacells}
\usepackage{listings}
\usepackage{svg}
\usepackage{xcolor, soul}
\usepackage{bm}
% \usepackage{minted}
\usepackage{braket}

% \usepackage{setspace}
% \usepackage{listings}
% \usepackage{listings}
% \usepackage[autoload=true]{jlcode}
% \usepackage{pygmentize}

\definecolor{cambridgeblue}{rgb}{0.64, 0.76, 0.68}

\sethlcolor{cambridgeblue}

\usepackage[margin=1.8cm]{geometry}
\newcommand{\C}{\mathbb C}
\newcommand{\D}{\bm D}
\newcommand{\R}{\mathbb R}
\newcommand{\Q}{\mathbb Q}
\newcommand{\Z}{\mathbb Z}
\newcommand{\N}{\mathbb N}
\newcommand{\PP}{\mathbb P}
\newcommand{\A}{\mathbb A}
\newcommand{\F}{\mathbb F}
\newcommand{\1}{\mathbf 1}
\newcommand{\ip}[1]{\left< #1 \right>}
\newcommand{\abs}[1]{\left| #1 \right|}
\newcommand{\norm}[1]{\left\| #1 \right\|}

\def\Tr{{\rm Tr}}
\def\tr{{\rm tr}}
\def\Var{{\rm Var}}
\def\calA{{\mathcal A}}
\def\calB{{\mathcal B}}
\def\calD{{\mathcal D}}
\def\calE{{\mathcal E}}
\def\calG{{\mathcal G}}
\def\from{{:}}
\def\lspan{{\rm span}}
\def\lrank{{\rm rank}}
\def\bd{{\rm bd}}
\def\acc{{\rm acc}}
\def\cl{{\rm cl}}
\def\sint{{\rm int}}
\def\ext{{\rm ext}}
\def\lnullity{{\rm nullity}}
\DeclareSIUnit\clight{\text{\ensuremath{c}}}
\DeclareSIUnit\fm{\femto\m}
\DeclareSIUnit\hplanck{\text{\ensuremath{h}}}


% \lstdefinelanguage{julia}%
%   {morekeywords={abstract,break,case,catch,const,continue,do,else,elseif,%
%       end,export,false,for,function,immutable,import,importall,if,in,%
%       macro,module,otherwise,quote,return,switch,true,try,type,typealias,%
%       using,while},%
%    sensitive=true,%
% %    alsoother={$},%
%    morecomment=[l]\#,%
%    morecomment=[n]{\#=}{=\#},%
%    morestring=[s]{"}{"},%
%    morestring=[m]{'}{'},%
% }[keywords,comments,strings]%

% \lstset{%
%     language         = Julia,
%     basicstyle       = \ttfamily,
%     keywordstyle     = \bfseries\color{blue},
%     stringstyle      = \color{magenta},
%     commentstyle     = \color{ForestGreen},
%     showstringspaces = false,
% }

% $
\begin{document}
\begin{center}
	\hrule
	\vspace{.4cm}
	{\textbf { \large CAS PY 452 --- Quantum Physics II}}
\end{center}
Emmy Blumenthal \hspace{\fill} \hspace{\fill}  \textbf{} Discussion Notes\  \\
\textbf{Date:}\  Sep 28, 2022   \hspace{\fill} \textbf{Email:}\ emmyb320@bu.edu \ 
\vspace{.4cm}
\hrule


\section*{Equivalence of first and second quantization for two particles and two orbitals}
\paragraph{First quantization:}

Consider the system where there are two orbitals $a,b$ and two particles.
The single particle basis states are $\psi_a(x), \psi_b(x)$, and the single-particle operator $\hat o(x)$ operates as,
\begin{align}
	\hat o(x) \psi_a(x)
	=
	o_{aa}\psi_a(x)
	+
	o_{ba}\psi_b(x),
	\qquad
	\hat o(x) \psi_b(x)
	=
	o_{ab}\psi_a(x)
	+
	o_{bb}\psi_b(x).
\end{align}
The values $o_{aa},o_{ba}, o_{ab},o_{bb}$ are the matrix elements of the {\em single-particle} operator $\hat o$, and we take them as a given.
The multi-particle basis states are,
\begin{align}
	\psi_{\alpha}(x_1,x_2)
	=
	\psi_a(x_1)
	\psi_a(x_2),
	\quad
	\psi_{\beta}(x_1,x_2)
	=
	\frac{1}{\sqrt{2}}
	\left(
	\psi_a(x_1)
	\psi_b(x_2)
	+
	\psi_a(x_2)
	\psi_b(x_1)
	\right)
	,
	\quad 
	\psi_\gamma(x_1,x_2)
	=
	\psi_b(x_1)
	\psi_b(x_2).
	\label{sqbasisstates}
\end{align}
In the first quantization, our goal is to calculate the matrix elements of the operator $\hat O = \hat o(x_1) + \hat o(x_2)$ in terms of the basis states $\psi_\alpha, \psi_\beta, \psi_\gamma$.
The notation $\hat o(x_1)$ means that this operator only acts on coordinate $x_1$ and leaves wave-functions in terms of $x_2$ alone.
In the prompt for problem 1, we want to find the matrix elements only corresponding to one basis state, so that's what we'll do here with the basis state $\psi_\alpha(x_1,x_2)$.
To find the matrix elements, we act on $\psi_\alpha(x_1,x_2)$ with the operator $\hat O$ and then express the resulting wave-function in terms of the basis states $\psi_\alpha, \psi_\beta, \psi_\gamma$:
\begin{align}
	\hat O [\psi_\alpha(x_1,x_2)]
	&=\nonumber
	\hat o(x_1) [\psi_\alpha(x_1,x_2)]
	+
	\hat o(x_2) [\psi_\alpha(x_1,x_2)]
	=
	\hat o(x_1) [\psi_a(x_1)\psi_a(x_2)]
	+
	\hat o(x_2) [\psi_a(x_1)\psi_a(x_2)]
	\\
	&=
	\psi_a(x_2) \hat o(x_1) [\psi_a(x_1)]
	+
	\psi_a(x_1)\hat o(x_2) [\psi_a(x_2)]
	\nonumber
	\\
	&=\nonumber
	\psi_a(x_2) (o_{aa}\psi_a(x_1)+o_{ba}\psi_b(x_1))
	+
	\psi_a(x_1)  (o_{aa}\psi_a(x_2)+o_{ba}\psi_b(x_2))
	\\
	&=\nonumber
	2o_{aa}\psi_a(x_1)\psi_a(x_2)+o_{ba}(\psi_a(x_2)\psi_b(x_1)
	+\psi_a(x_1)\psi_b(x_2))
	\\
	&=
	2o_{aa}\psi_\alpha(x_1,x_2)
	+
	\sqrt{2}
	o_{ba}\psi_\beta(x_1,x_2)
	+
	0\psi_\gamma(x_1,x_2).
\end{align}
In the last line, we factored our expression in terms of the basis states $\psi_\alpha, \psi_\beta, \psi_\gamma$
The matrix elements in this multi-particle basis are then $2o_{aa}$, $\sqrt{2}o_{ba}$, $0$.
We could write this as a column in a matrix like so:
\begin{align}
	\left(
		\begin{array}{ccc}
			2 o_{aa} & \cdot & \cdot\\
			\sqrt{2} o_{ba} & \cdot & \cdot\\
			0 & \cdot & \cdot\\
		\end{array}
	\right).
\end{align}
Here, the dots are the matrix elements that we do not know yet; we would find these matrix elements by computing $\hat O \psi_\beta$ and $\hat O \psi_\gamma$ and then factoring the results in terms of $\psi_\alpha, \psi_\beta, \psi_\gamma$.
In the problem prompt, you are only asked to find one column of this matrix: the column corresponding to acting with $\hat O$ on the multi-particle basis state with two particles in orbital $a$, one particle in orbital $b$, and no particles in orbital $c$.
One of the steps in this problem is finding expressions for the multi-particle basis states (similar to what happened in line \ref{sqbasisstates}).
Note that we are working with bosons here and in problem 1 of the homework.

\paragraph{Second quantization:}

Now, we work in the second-quantized notation, so our basis states are $\ket{2,0}$, $\ket{1,1}$, and $\ket{0,2}$.
The operator $\hat O$ is expressed as,
\begin{align}
	\hat O =
	\hat a_a^\dagger
	o_{aa}
	\hat a_a
	+
	\hat a_b^\dagger
	o_{ba}
	\hat a_a
	+
	\hat a_a^\dagger
	o_{ab}
	\hat a_b
	+
	\hat a_b^\dagger
	o_{bb}
	\hat a_b,
\end{align}
in the second quantization.
To compute matrix elements, we use the rules $\hat a \ket{n} = \sqrt{n} \ket{n-1}$, $\hat a^\dagger \ket{n} = \sqrt{n+1} \ket{n+1}$, and $\hat a\ket{0} = 0$.
Similarly to before, we act on $\ket{2,0}$ with $\hat O$ and then factor in terms of basis states,
\begin{align}
	\hat O \ket{2,0}
	&=\nonumber
	\hat a_a^\dagger
	o_{aa}
	\hat a_a
	\ket{2,0}
	+
	\hat a_b^\dagger
	o_{ba}
	\hat a_a
	\ket{2,0}
	+
	\hat a_a^\dagger
	o_{ab}
	\hat a_b
	\ket{2,0}
	+
	\hat a_b^\dagger
	o_{bb}
	\hat a_b
	\ket{2,0}
	=
	\sqrt{2}
	o_{aa}
	\hat a_a^\dagger
	\ket{1,0}
	+
	\sqrt{2}
	o_{ba}
	\hat a_b^\dagger
	\ket{1,0}
	+
	0
	+
	0
	\\
	&=
	\sqrt{2} \times \sqrt{2} o_{aa} \ket{2,0}
	+
	\sqrt{2}
	o_{ba}
	\ket{1,1}
	+
	0 \ket{0,2}.
\end{align}
Here we see that we are able to re-produce the same matrix elements from the first quantization where the basis states $\ket{2,0}, \ket{1,1},\ket{0,2}$ correspond to $\psi_\alpha,\psi_\beta,\psi_\gamma$.
In problem 1 of the homework, you will follow this same procedure to show the equivalence of the first and second quantized methods but with three orbitals and three particles.



\end{document}





