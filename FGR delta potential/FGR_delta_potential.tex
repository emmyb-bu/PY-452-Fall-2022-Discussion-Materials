\documentclass[10pt]{article}
% \usepackage{geometry}
% \geometry{margin=0.2in}
\usepackage[utf8]{inputenc}

\nonstopmode
% \usepackage{minted}[cache=false]
\usepackage{graphicx} % Required for including pictures
\usepackage[figurename=Figure]{caption}
% \usepackage{float}    % For tables and other floats
\usepackage{amsmath}  % For math
\usepackage{amssymb}  % For more math
\usepackage{fullpage} % Set margins and place page numbers at bottom center
% \usepackage{paralist} % paragraph spacing
% \usepackage{subfig}   % For subfigures
%\usepackage{physics}  % for simplified dv, and 
% \usepackage{enumitem} % useful for itemization
% \usepackage{siunitx}  % standardization of si units
\usepackage{hyperref}
% \usepackage{mmacells}
% \usepackage{listings}
% \usepackage{svg}
% \usepackage{xcolor, soul}
% \usepackage{bm}
\usepackage{braket}
% \usepackage{cancel}
% \usepackage{setspace}
% \usepackage{listings}
% \usepackage{listings}
% \usepackage[autoload=true]{jlcode}
% \usepackage{pygmentize}

% \definecolor{cambridgeblue}{rgb}{0.64, 0.76, 0.68}

% \sethlcolor{cambridgeblue}

\usepackage[margin=1.8cm]{geometry}
\newcommand{\C}{\mathbb C}
\newcommand{\D}{\bm D}
\newcommand{\R}{\mathbb R}
\newcommand{\Q}{\mathbb Q}
\newcommand{\Z}{\mathbb Z}
\newcommand{\N}{\mathbb N}
\newcommand{\PP}{\mathbb P}
\newcommand{\A}{\mathbb A}
\newcommand{\F}{\mathbb F}
\newcommand{\1}{\mathbf 1}
\newcommand{\ip}[1]{\left< #1 \right>}
\newcommand{\abs}[1]{\left| #1 \right|}
\newcommand{\norm}[1]{\left\| #1 \right\|}

\def\Tr{{\rm Tr}}
\def\tr{{\rm tr}}
\def\Var{{\rm Var}}
\def\calA{{\mathcal A}}
\def\calB{{\mathcal B}}
\def\calD{{\mathcal D}}
\def\calE{{\mathcal E}}
\def\calG{{\mathcal G}}
\def\from{{:}}
\def\lspan{{\rm span}}
\def\lrank{{\rm rank}}
\def\bd{{\rm bd}}
\def\acc{{\rm acc}}
\def\cl{{\rm cl}}
\def\sint{{\rm int}}
\def\ext{{\rm ext}}
\def\lnullity{{\rm nullity}}
% \DeclareSIUnit\clight{\text{\ensuremath{c}}}
% \DeclareSIUnit\fm{\femto\m}
% \DeclareSIUnit\hplanck{\text{\ensuremath{h}}}


% \lstdefinelanguage{julia}%
%   {morekeywords={abstract,break,case,catch,const,continue,do,else,elseif,%
%       end,export,false,for,function,immutable,import,importall,if,in,%
%       macro,module,otherwise,quote,return,switch,true,try,type,typealias,%
%       using,while},%
%    sensitive=true,%
% %    alsoother={$},%
%    morecomment=[l]\#,%
%    morecomment=[n]{\#=}{=\#},%
%    morestring=[s]{"}{"},%
%    morestring=[m]{'}{'},%
% }[keywords,comments,strings]%

% \lstset{%
%     language         = Julia,
%     basicstyle       = \ttfamily,
%     keywordstyle     = \bfseries\color{blue},
%     stringstyle      = \color{magenta},
%     commentstyle     = \color{ForestGreen},
%     showstringspaces = false,
% }

% $
\begin{document}
\begin{center}
	\hrule
	\vspace{.4cm}
	{\textbf { \large CAS PY 452 --- Quantum Physics II}}
\end{center}
Emmy Blumenthal \hspace{\fill} \hspace{\fill}  \textbf{} Discussion Notes\  \\
\textbf{Date:}\  December 7, 2022   \hspace{\fill} \textbf{Email:}\ emmyb320@bu.edu

\vspace{.4cm}
\hrule

\section*{Exciting a particle in a $\delta$-potential}

\paragraph{Problem:}

Consider a particle in a bound 1D $\delta$-potential with Hamiltonian:
\begin{align}
	\hat H_0 = \frac{\hat p^2}{2m} - g \delta(x),
\end{align}
where $g> 0$.
Compute the energy absorption rate of the particle if it is then perturbed by a standing plane wave with amplitude $\mathcal{E}(t)$
\begin{align}
	\hat H' = q \mathcal{E}(t) \sin(\xi \hat x )
\end{align}
Assume that $q \mathcal{E}(t) \ll m g^2/\hbar^2 $ so that we may apply perturbation theory.\footnote{Here, $g$ has units of [energy]$\times$[length], $\mathcal{E}(t)$ has units of [energy]$\times$/[charge], and $q$ has units of [charge].}

\paragraph{Reminder:}

First, recall that there is only one bound state for the $\delta$-potential:
\begin{align}
	\psi^\delta(x) = \sqrt{\kappa}e^{-\kappa|x|}, \qquad \kappa = \frac{mg}{\hbar^2},
\end{align}
which has energy,
\begin{align}
	E^\delta =- \frac{\hbar^2 \kappa^2}{2 m} = - \frac{m g^2}{2 \hbar^2}.
\end{align}
Next, recall that the `particle in a box' with Hamiltonian $\hat H^\square = \frac{\hat p^2}{2m} + V(x)$, where $V(x) = \begin{cases}
	0 & |x| \leq L/2\\
	\infty & |x| > L/2
\end{cases}$,
has wave-functions,
\begin{align}
	\psi_n^\square(x) = \sqrt{\frac{2}{L}} 
	\begin{cases}
		\sin(k_n x) & \text{even $n$},\\
		\cos(k_n x) & \text{odd $n$},
	\end{cases}
	\qquad 
	k_n = \frac{n\pi}{L},
\end{align}
and energies,
\begin{align}
	E_n^\square = \frac{\hbar^2  k_n^2}{2m} =\frac{n^2 \pi^2\hbar^2}{2m L^2}.
\end{align}


\paragraph{Solution:}

To evaluate transition rates from the bound state to excited states, we need to compute matrix elements of $\hat H'$ between the initial state $\psi^\delta$ and the final state as well as the density of excited states.
Our final states are going to be plane waves, but we don't have a good way to calculate density of states for free-particle plane waves.
We will make an approximation by considering the $\delta$-potential to be centered in the middle of a large particle-in-a-box potential.
To make this work, our bound-state wave-function needs to satisfy boundary conditions $\psi(\pm L/2) = 0$; if we assume $L \gg 1/\kappa$, then $\psi^\delta$ approximately fits these boundary conditions, so we will still take $\psi^\delta$ to be the bound ground state.
Using these approximations, we will consider transitions between $\psi^\delta$ to $\psi^\square_n$.

First, we compute the matrix elements,
\begin{align}
	\ip{\psi_n^\square | \hat H' | \psi^\delta}
	=
	q \mathcal{E}(t)
	\int_{-\infty}^\infty
	\psi_n^\square(x)
	\cos(\xi x)
	\psi^\delta(x)
	dx
	=
	q \mathcal{E}(t)
	\sqrt{\kappa}
	% \sqrt{\frac{2\kappa}{L}}
	\int_{-\infty}^\infty
	\psi_n^\square(x)
	\sin(\xi x )
	e^{-\kappa|x|}
	dx.
\end{align}
Notice that if $n$ is odd, then $\psi_n^\square(x)$ is even and the integrand is odd, meaning that $\ip{\psi_n^\square | \hat H' | \psi^\delta} = 0$.
Therefore, only the matrix elements where $n$ is even remain:
\footnote{
	Technically, the limits of integration should be $\pm L/2$, but we will end up taking a limit $L \to \infty$ at the end, and choosing these limits here will not effect these calculations as other factors of $L$ will cancel out.
}
\begin{align}
	\ip{\psi_n^\square | \hat H' | \psi^\delta}
	&=
	q \mathcal{E}(t)
	\sqrt{\frac{2\kappa}{L}}
	\int_{-\infty}^\infty
	\sin(k_n x)
	\sin(\xi x )
	e^{-\kappa|x|}
	dx
	=
	q \mathcal{E}(t)
	\sqrt{\frac{2\kappa}{L}}
	\left(\frac{i}{2}\right)^2
	\int_{-\infty}^\infty
	\left(
		e^{- i k_n x}
		-
		e^{ i k_n x}
	\right)
	\left(
		e^{- i \xi x}
		-
		e^{ i \xi x}
	\right)
	e^{-\kappa|x|}
	dx\nonumber
	\\
	&=
	-
	q \mathcal{E}(t)
	\frac{1}{2}
	\sqrt{\frac{\kappa}{2L}}
	\int_{-\infty}^\infty
	\left(
		e^{- i (k_n + \xi) x - \kappa |x|}
		-
		e^{- i (k_n - \xi) x - \kappa |x|}
		-
		e^{- i (-k_n + \xi) x - \kappa |x|}
		+
		e^{- i (-k_n - \xi) x - \kappa |x|}
	\right)
	dx.
\end{align}
There are four integrals here, but we can find them by computing just one then substituting:
\begin{align}
	\int_{-\infty}^\infty
	e^{-i \alpha x - \kappa |x|} dx
	&=\nonumber
	\int_{-\infty}^0
	e^{- i\alpha x + \kappa x} dx
	+
	\int_{0}^\infty
	e^{-i \alpha x - \kappa x} dx
	=
	\int_{0}^\infty
	e^{ -(\kappa - i\alpha ) x} dx
	+
	\int_{0}^\infty
	e^{- (\kappa +i \alpha) x} dx
	\\
	&=
	\frac{1}{\kappa - i\alpha}
	+
	\frac{1}{\kappa + i\alpha}
	=
	\frac{2\kappa}{\alpha^2+\kappa^2}.
\end{align}
Now, we substitute $\alpha = k_n +\xi, k_n - \xi, -k_n + \xi, -k_n - \xi$ to find,
\begin{align}
	\ip{\psi_n^\square | \hat H' | \psi^\delta}
	&=
	-
	q \mathcal{E}(t)
	\sqrt{\frac{\kappa}{2L}}
	\left(
		\frac{\kappa}{(k_n + \xi)^2 + \kappa^2}
		-
		\frac{\kappa}{(k_n - \xi)^2 + \kappa^2}
		-
		\frac{\kappa}{(-k_n + \xi)^2 + \kappa^2}
		+
		\frac{\kappa}{(-k_n - \xi)^2 + \kappa^2}
	\right)
	\nonumber\\
	&=
	q \mathcal{E}(t)
	\sqrt{\frac{2}{L\kappa}}
	\left(
		\frac{\kappa^2}{(k_n - \xi)^2 + \kappa^2}
		-
		\frac{\kappa^2}{(k_n + \xi)^2 + \kappa^2}
	\right),
	\qquad \left(
		n = 2,4,6,8,\dots
	\right).
\end{align}
Note that we can use $k_n = \sqrt{2m E_n^\square / \hbar^2}$ to write these matrix elements in terms of the energy of $\psi_n^\square$.
The next ingredient we need to use the Fermi golden rule is the density of states to which the particle is going to transition.
For this, we will be working with the continuum limit where $L \to \infty$, so the spacing between $k_n = n\pi/L$ becomes very small and $k_n$ becomes a continuous variable.
The density of states for a particle in a box is,
\begin{align}
	\rho^\square(E)
	=
	\left(\frac{d E^\square}{dn}\right)^{-1}
	=
	\left(\frac{d E^\square}{dk}
	\frac{dk}{dn}\right)^{-1}
	=
	\left(
		\frac{\hbar^2 k_n}{m}
		\frac{\pi}{L}
	\right)^{-1}
	=
	\frac{m L}{\hbar^2 k_n \pi}
	=
	\left(
		\frac{\pi\hbar^2}{mL}\frac{\sqrt{2m E}}{\hbar}
	\right)^{-1}
	=
	\frac{L}{\pi \hbar}\sqrt{\frac{m}{2 E}}
	,
\end{align}
where we have used $E_n^\square = \frac{\hbar^2 k_n^2}{2m} \implies k_n = \sqrt{2m E_n^\square/\hbar^2}$.
However, for our purposes, when $L \to \infty$ (i.e., we take the continuum limit), only half the states in the particle in a box within some small interval $(E, E+\delta E)$ are actually accessible because $\hat H'$ has zero matrix elements with all odd $n$, so,
\begin{align}
	\rho(E)
	=
	\frac{1}{2}\rho^\square (E)
	=
	\frac{L}{2\pi\hbar}
	\sqrt{\frac{m}{2E}}
	=
	\frac{m L}{2\hbar^2 k \pi}
	.
\end{align}
With all this in place, we can use the Fermi golden rule to evaluate transition rates:
\begin{align}
	\Gamma_{\delta \to E}
	&=\nonumber
	\frac{2\pi}{\hbar}
	\left|
		\bra{\psi_E^\square} \hat H' \ket{\psi^\delta}
	\right|^2
	\rho(E)
	=
	\frac{2\pi}{\hbar}
	\left[
		q \mathcal{E}(t)
		\sqrt{\frac{2}{L\kappa}}
		\left(
			\frac{\kappa^2}{(k - \xi)^2 + \kappa^2}
			-
			\frac{\kappa^2}{(k + \xi)^2 + \kappa^2}
		\right)
	\right]^2
	\frac{m L}{2 \hbar^2 k \pi}
	\\
	&=
	q^2 \mathcal{E}(t)^2
	\frac{2m}{\hbar^3 \kappa k }
		\left(
			\frac{1}{\left(\frac{k - \xi}{\kappa}\right)^2 + 1}
			-
			% \frac{\kappa^2}{(k + \xi)^2 + \kappa^2}
			\frac{1}{\left(\frac{k + \xi}{\kappa}\right)^2 + 1}
		\right)^2
		\\
	&=
	\frac{q^2 \mathcal{E}(t)^2}{\hbar^2 \kappa }
	\sqrt{\frac{2m}{E}}
		\left(
			\frac{1}{\left(\frac{\sqrt{2m E/\hbar^2} - \xi}{\kappa}\right)^2 + 1}
			-
			% \frac{\kappa^2}{\left(\sqrt{2m E/\hbar^2} + \xi\right)^2 + \kappa^2}
			\frac{1}{\left(\frac{\sqrt{2m E/\hbar^2} + \xi}{\kappa}\right)^2 + 1}
		\right)^2.
\end{align}
It is important to note that transitions to only certain energies will be allowed, consistent with how energy enters and leaves the system due to the time-dependent variation of the strength of the perturbation which is controlled by $\mathcal{E}(t)$.
In particular, the energy of the system can only change by $\pm \hbar \omega$ where $\omega$ is the frequency of the perturbation $\mathcal{E}(t)$.
For example if $\mathcal{E}(t) = \mathcal{E}_0 \cos(\omega t)$, then the energy of the system can only change by $\pm\hbar \omega$, so the transition rates will be:
\begin{align}
	\Gamma_{\delta \to E}
	=
	\frac{q^2 \mathcal{E}(t)^2}{\hbar^2 \kappa }
	\sqrt{\frac{2m}{E}}
		\left(
			\frac{1}{\left(\frac{\sqrt{2m E/\hbar^2} - \xi}{\kappa}\right)^2 + 1}
			-
			% \frac{\kappa^2}{\left(\sqrt{2m E/\hbar^2} + \xi\right)^2 + \kappa^2}
			\frac{1}{\left(\frac{\sqrt{2m E/\hbar^2} + \xi}{\kappa}\right)^2 + 1}
		\right)^2
		\delta(E - E^\delta \pm \hbar \omega).
\end{align}




\end{document}






