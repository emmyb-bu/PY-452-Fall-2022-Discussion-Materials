\documentclass[10pt]{article}
% \usepackage{geometry}
% \geometry{margin=0.2in}
\usepackage[utf8]{inputenc}

\nonstopmode
% \usepackage{minted}[cache=false]
\usepackage{graphicx} % Required for including pictures
\usepackage[figurename=Figure]{caption}
% \usepackage{float}    % For tables and other floats
\usepackage{amsmath}  % For math
\usepackage{amssymb}  % For more math
\usepackage{fullpage} % Set margins and place page numbers at bottom center
% \usepackage{paralist} % paragraph spacing
% \usepackage{subfig}   % For subfigures
%\usepackage{physics}  % for simplified dv, and 
% \usepackage{enumitem} % useful for itemization
\usepackage{siunitx}  % standardization of si units
\usepackage{hyperref}
\usepackage{mmacells}
% \usepackage{listings}
% \usepackage{svg}
% \usepackage{xcolor, soul}
\usepackage{bm}
% \usepackage{minted}
\usepackage{braket}

% \usepackage{setspace}
% \usepackage{listings}
% \usepackage{listings}
% \usepackage[autoload=true]{jlcode}
% \usepackage{pygmentize}

% \definecolor{cambridgeblue}{rgb}{0.64, 0.76, 0.68}

% \sethlcolor{cambridgeblue}

\usepackage[margin=1.8cm]{geometry}
\newcommand{\C}{\mathbb C}
\newcommand{\D}{\bm D}
\newcommand{\R}{\mathbb R}
\newcommand{\Q}{\mathbb Q}
\newcommand{\Z}{\mathbb Z}
\newcommand{\N}{\mathbb N}
\newcommand{\PP}{\mathbb P}
\newcommand{\A}{\mathbb A}
\newcommand{\F}{\mathbb F}
\newcommand{\1}{\mathbf 1}
\newcommand{\ip}[1]{\left< #1 \right>}
\newcommand{\abs}[1]{\left| #1 \right|}
\newcommand{\norm}[1]{\left\| #1 \right\|}

\def\Tr{{\rm Tr}}
\def\tr{{\rm tr}}
\def\Var{{\rm Var}}
\def\calA{{\mathcal A}}
\def\calB{{\mathcal B}}
\def\calD{{\mathcal D}}
\def\calE{{\mathcal E}}
\def\calG{{\mathcal G}}
\def\from{{:}}
\def\lspan{{\rm span}}
\def\lrank{{\rm rank}}
\def\bd{{\rm bd}}
\def\acc{{\rm acc}}
\def\cl{{\rm cl}}
\def\sint{{\rm int}}
\def\ext{{\rm ext}}
\def\lnullity{{\rm nullity}}
\DeclareSIUnit\clight{\text{\ensuremath{c}}}
\DeclareSIUnit\fm{\femto\m}
\DeclareSIUnit\hplanck{\text{\ensuremath{h}}}


% \lstdefinelanguage{julia}%
%   {morekeywords={abstract,break,case,catch,const,continue,do,else,elseif,%
%       end,export,false,for,function,immutable,import,importall,if,in,%
%       macro,module,otherwise,quote,return,switch,true,try,type,typealias,%
%       using,while},%
%    sensitive=true,%
% %    alsoother={$},%
%    morecomment=[l]\#,%
%    morecomment=[n]{\#=}{=\#},%
%    morestring=[s]{"}{"},%
%    morestring=[m]{'}{'},%
% }[keywords,comments,strings]%

% \lstset{%
%     language         = Julia,
%     basicstyle       = \ttfamily,
%     keywordstyle     = \bfseries\color{blue},
%     stringstyle      = \color{magenta},
%     commentstyle     = \color{ForestGreen},
%     showstringspaces = false,
% }

% $
\begin{document}
\begin{center}
	\hrule
	\vspace{.4cm}
	{\textbf { \large CAS PY 452 --- Quantum Physics II}}
\end{center}
Emmy Blumenthal \hspace{\fill} \hspace{\fill}  \textbf{} Discussion Notes\  \\
\textbf{Date:}\  Oct 5, 2022   \hspace{\fill} \textbf{Email:}\ emmyb320@bu.edu \ 
\vspace{.4cm}
\hrule

\section*{Practice perturbing the quantum harmonic oscillator}

\paragraph{Problem statement:}
Consider a one-dimensional quantum harmonic oscillator; that is, consider a particle with the following Hamiltonian:
\begin{align}
	\hat H
	=
	\frac{1}{2m} \hat p^2
	+
	\frac{1}{2} m \omega^2 \hat x^2.
\end{align}
Now, perturb the Hamiltonian as $\hat H \to \hat H + \hat H'$ where,
\begin{align}
	\hat H'
	=
	\epsilon \hat x.
\end{align}
Solve for eigenstates and energies exactly and using time-independent perturbation theory to second order for the energies and first order for the states.

\paragraph{A useful reminder:}

The (unperturbed) Harmonic oscillator energy eigenstates, $\ket{\psi_n^{(0)}}$ in the position representation are,
\begin{align}
	\psi_n^{(0)}
	=
	\frac{1}{\sqrt{2^n}n!}
	\left(\frac{m \omega}{\hbar \pi}\right)^{1/4}
	e^{-
	\frac{m \omega}{\hbar}x^2/2
	}
	H_n\left(
		x\sqrt{\frac{m\omega}{\hbar}}
	\right),
\end{align}
and the (unperturbed) energy eigenvalues are,
\begin{align}
	E_n^{(0)}
	=
	\hbar \omega \left(
		n + \frac{1}{2}
	\right).
\end{align}
We can simplify computations using ladder operators,
\begin{align}
	\hat a
	=
	\sqrt{\frac{m \omega}{2 \hbar}} ( \hat x + \frac{i}{m\omega}\hat p),
	\qquad
	\hat a^\dagger
	=
	\sqrt{\frac{m \omega}{2 \hbar}} ( \hat x - \frac{i}{m\omega}\hat p),
\end{align}
which obey the relations,\footnote{There was previously a typo in line \ref{a-type}; $\hat p$ had the wrong units! It is now corrected. My apologies.}
\begin{align}\label{a-type}
	\hat x = \sqrt{
		\frac{\hbar}{2m \omega}
	}
	(\hat a^\dagger +\hat a),
	\qquad
	\hat p =i \sqrt{
		\frac{\hbar m \omega}{2}
		% \frac{\hbar}{2m \omega}
	}
	(\hat a^\dagger - \hat a),
\end{align}
and act on the energy eigenstates as,
\begin{align}
	\hat a \ket{\psi_n^{(0)}}
	=
	\sqrt{n}\ket{\psi_{n-1}^{(0)}},
	\qquad 
	\hat a^\dagger \ket{\psi_n^{(0)}}
	=
	\sqrt{n+1}\ket{\psi_{n+1}^{(0)}},
	\qquad
	\hat a \ket{\psi_0^{(0)}}
	=
	0.
\end{align}
This saves us from having to do integrals like this:
\begin{align}
	\braket{\psi_0^{(0)} | \hat x|\psi_{1}^{0}}
	=
	\int_{-\infty}^\infty
	dx
	\,
	\psi_0^{(0)}(x)^\ast
	x
	\psi_1^{(0)}(x)
	=
	\left(\frac{m \omega}{\hbar \pi}\right)^{1/4}
	\sqrt{2\pi}
	\left(\frac{m \omega}{\hbar \pi}\right)^{3/4}
	\int_{-\infty}^\infty
	dx\,
	x
	e^{-
	\frac{m \omega}{2\hbar}x^2
	}
	x
	e^{-
	\frac{m \omega}{2\hbar}x^2
	}
	=
	\sqrt{\frac{\hbar}{2m\omega}},
\end{align}
which I actually did in {\em Mathematica}:
{\small
\begin{mmaCell}[morefunctionlocal={x}]{Input}
\mmaSqrt{2\mmaDef{\(\pmb{\pi}\)}}\mmaSup{(\mmaFrac{m \mmaUnd{\(\pmb{\omega}\)}}{\mmaUnd{\(\pmb{\hbar}\)} \mmaDef{\(\pmb{\pi}\)}})}{1/4} \mmaSup{(\mmaFrac{m \mmaUnd{\(\pmb{\omega}\)}}{\mmaUnd{\(\pmb{\hbar}\)} \mmaDef{\(\pmb{\pi}\)}})}{3/4}Integrate[ x \mmaSup{\mmaDef{e}}{-\mmaFrac{m \mmaSup{x}{2} \mmaUnd{\(\pmb{\omega}\)}}{2 \mmaUnd{\(\pmb{\hbar}\)}}}  x \mmaSup{\mmaDef{e}}{-\mmaFrac{m \mmaSup{x}{2} \mmaUnd{\(\pmb{\omega}\)}}{2 \mmaUnd{\(\pmb{\hbar}\)}}} ,\{x,-\mmaDef{\(\pmb{\infty}\)},\mmaDef{\(\pmb{\infty}\)}\},Assumptions->\mmaUnd{\(\pmb{\omega}\)}>0&&\mmaUnd{\(\pmb{\hbar}\)}>0&&m>0]
\end{mmaCell}
\begin{mmaCell}{Output}
\mmaSqrt{\mmaFrac{\mmaUnd{\(\pmb{\hbar}\)}}{2 m \mmaUnd{\(\pmb{\omega}\)}}}
\end{mmaCell}
}
\noindent
Instead, we do this:
\begin{align}
	\braket{\psi_0^{(0)} | \hat x|\psi_{1}^{0}}
	&=\nonumber
	\braket{\psi_0^{(0)} | \hat x|\psi_{1}^{0}}
	=
	\braket{\psi_0^{(0)}|
	\sqrt{\frac{\hbar}{2m\omega}}
	(\hat a^\dagger + \hat a)
	|
	\psi_1^{(0)}
	}
	=
	\sqrt{\frac{\hbar}{2m\omega}}
	\bra{\psi_0^{(0)}}\left(
		\hat a^\dagger\ket{\psi_1^{(0)}} + \hat a\ket{\psi_1^{(0)}}
	\right)\\
	&=\nonumber
	\sqrt{\frac{\hbar}{2m\omega}}
	\bra{\psi_0^{(0)}}\left(
		\sqrt{1+1}\ket{\psi_{1+1}^{(0)}} + \sqrt{1}\ket{\psi_{1-1}^{(0)}}
	\right)
	=
	\sqrt{\frac{\hbar}{2m\omega}}
	\left(
		\sqrt{2}
		\braket{\psi_0^{(0)}|\psi_2^{(0)}}
		+
		\braket{\psi_0^{(0)}|\psi_0^{(0)}}
	\right)\\
	&=
	\sqrt{\frac{\hbar}{2m\omega}}
	(\sqrt{2} \times 0 + 1)
	=
	\sqrt{\frac{\hbar}{2m\omega}},
\end{align}
which agrees with the integral.
Note that in the last line, we used that the energy eigenstates form an orthonormal basis,
\begin{align}
	\braket{\psi_n^{(0)}|\psi_m^{(0)}}
	=
	\delta_{n,m},
\end{align}
where $\delta_{nm}$ is the Kronecker delta symbol.
{\bf At any point in this computation, we could replace one of the bra-ket expressions with an integral and obtain the same results.}
\footnote{
	There is something tricky in notation here: $m$ is the symbol for both an index and the mass.
	It should hopefully be clear from context which is which, but be aware that this notational clash exists.
}
% $\hat a \ket{\psi^{(0)}_n} = \sqrt{n} \ket{\psi^{(0)}_{n-1}}$ and $\hat a^\dagger \ket{\psi^{(0)}_n} = \sqrt{n+1} \ket{\psi^{(0)}_{n+1}}$

\paragraph{Perturbative solution:}
Write the Hamiltonian as $\hat H = \hbar \omega (\hat a^\dagger \hat a + \frac{1}{2})$ and the perturbing Hamiltonian as $\hat H' = \epsilon\sqrt{\hbar / (2m \omega)} (\hat a^\dagger + \hat a)$.
We compute the perturbed energies as:
\begin{align}
	E^{(1)}_n
	&=\nonumber
	\braket{\psi_n^{(0)}|\hat H'| \psi_n^{(0)}}
	=
	\epsilon
	\sqrt{\frac{\hbar}{2m\omega}}
	\bra{\psi_n^{(0)}}
	(\hat a + \hat a^\dagger)
	\ket{\psi_n^{(0)}}
	=
	\epsilon
	\sqrt{\frac{\hbar}{2m\omega}}
	\bra{\psi_n^{(0)}}
	(\hat a + \hat a^\dagger)
	\ket{\psi_n^{(0)}}
	\\
	&=
	\epsilon
	\sqrt{\frac{\hbar}{2m\omega}}
	\bra{\psi_n^{(0)}}
	(\sqrt{n} \ket{\psi_{n-1}^{(0)}}+ \sqrt{n+1}\ket{\psi_{n+1}^{(0)}})
	=
	\epsilon
	\sqrt{\frac{\hbar}{2m\omega}}
	(\sqrt{n} \delta_{n,n-1}+ \sqrt{n+1}
	\delta_{n,n+1}
	=
	0.
\end{align}
% \begin{align}
% 	E^{(1)}_n
% 	&=\nonumber
% 	\bra{\psi_n^{(0)}} \hat H' \ket{\psi_n^{(0)}}
% 	=
% 	\epsilon
% 	\bra{\psi_n^{(0)}}\hat x\ket{\psi_n^{(0)}}
% 	=
% 	\epsilon
% 	\sqrt{\hbar /(2m \omega)}
% 	\braket{\psi_n^{(0)}|a^\dagger + a|\psi_n^{(0)}}
% 	\\
% 	&=
% 	\epsilon
% 	\sqrt{\hbar /(2m \omega)}
% 	(\sqrt{n+1}\braket{\psi_n^{(0)}|\psi_{n+1}^{(0)}} + \sqrt{n} \braket{\psi_n^{(0)}|\psi_{n-1}^{(0)}}
% 	=
% 	0.
% \end{align}
The first-order corrections are zero.
The first-order corrections to the stationary states are,
\begin{align}
	\ket{\psi_n^{(1)}}
	&=\nonumber
	\sum_{m \ne n}
	\frac{\braket{\psi_m^{(0)} | \hat H' | \psi_n^{(0)}}}{E_n^{(0)} - E_m^{(0)}}
	\ket{\psi_m^{(0)}}
	=
	\epsilon\sqrt{\frac{\hbar}{2m\omega}}
	\sum_{m \ne n}
	\frac{\braket{\psi_m^{(0)} | \hat a^\dagger + \hat a | \psi_n^{(0)}}}{
		% E_n^{(0)} - E_m^{(0)}
		\hbar \omega(\frac{1}{2} + n - \frac{1}{2} - m)
		% E_n^{(0)} - E_m^{(0)}
	}
	\ket{\psi_m^{(0)}}
	\\
	&=\nonumber
	\epsilon\sqrt{\frac{\hbar}{2m\omega}}
	\sum_{m \ne n}
	\frac{
		\sqrt{n+1}\braket{\psi_m^{(0)} | \psi_{n+1}^{(0)}}
		% \braket{\psi_n^{(0)} | \hat a^\dagger + \hat a | \psi_m^{(0)}}
		+
		\sqrt{n}\braket{\psi_m^{(0)} | \psi_{n-1}^{(0)}}
		}{
		% E_n^{(0)} - E_m^{(0)}
		n-m
		% E_n^{(0)} - E_m^{(0)}
	}
	\ket{\psi_m^{(0)}}
	=
	\epsilon\sqrt{\frac{\hbar}{2m\omega}}
	\sum_{m \ne n}
	\frac{
		\sqrt{n+1}\delta_{m,n+1}
		% \braket{\psi_n^{(0)} | \hat a^\dagger + \hat a | \psi_m^{(0)}}
		+
		\sqrt{n}
		\delta_{m,n-1}
		% \braket{\psi_m^{(0)} | \psi_{n-1}^{(0)}}
		}{
		% E_n^{(0)} - E_m^{(0)}
		n-m
		% E_n^{(0)} - E_m^{(0)}
	}
	\ket{\psi_m^{(0)}}
	\\
	&=
	\epsilon\sqrt{\frac{\hbar}{2m\omega}}
	\left[
	0
	+
	\cdots
	+
	0
	+
	\frac{
		\sqrt{n+1}\delta_{n-1,n+1}
		% \braket{\psi_n^{(0)} | \hat a^\dagger + \hat a | \psi_m^{(0)}}
		+
		\sqrt{n}
		\delta_{n-1,n-1}
		% \braket{\psi_m^{(0)} | \psi_{n-1}^{(0)}}
		}{
		% E_n^{(0)} - E_m^{(0)}
		n-(n-1)
		% E_n^{(0)} - E_m^{(0)}
	}
	\ket{\psi_{n-1}^{(0)}}
	+
	\frac{
		\sqrt{n+1}\delta_{n+1,n+1}
		% \braket{\psi_n^{(0)} | \hat a^\dagger + \hat a | \psi_m^{(0)}}
		+
		\sqrt{n}
		\delta_{n+1,n-1}
		% \braket{\psi_m^{(0)} | \psi_{n-1}^{(0)}}
		}{
		% E_n^{(0)} - E_m^{(0)}
		n-(n+1)
		% E_n^{(0)} - E_m^{(0)}
	}
	\ket{\psi_{n+1}^{(0)}}
	+
	0
	+
	\cdots
	+
	0
	\right]
	\nonumber
	\\
	&=
	\epsilon\sqrt{\frac{\hbar}{2m\omega}}
	\left(
		\sqrt{n} \ket{\psi_{n-1}^{(0)}}
		-
		\sqrt{n+1} \ket{\psi_{n+1}^{(0)}}
	\right).
\end{align}
This makes our first-order perturbed energy eigenstates:
\begin{align}
	\ket{\psi_n^{\text{1st order}}}
	% \overset{!}{=}
	=
	\ket{\psi_n^{(0)}}
	+
	\ket{\psi_n^{(1)}}
	=
	\ket{\psi_n^{(0)}}
	+
	\epsilon\sqrt{\frac{\hbar}{2m\omega}}
	\left(
		\sqrt{n} \ket{\psi_{n-1}^{(0)}}
		-
		\sqrt{n+1} \ket{\psi_{n+1}^{(0)}}
	\right).
\end{align}
% But wait!!! This is not normalized (observe the exclamation point over the equals sign)! The normalized stated is,
% \begin{align}
% 	\ket{\psi_n^{\text{1st order}}}
% 	=
% 	\frac{1}{\sqrt{
% 		1 + \frac{\epsilon^2 \hbar}{2 m \omega}(2n +1)
% 	}}
% 	\left[
% 	\ket{\psi_n^{(0)}}
% 	+
% 	\epsilon\sqrt{\frac{\hbar}{2m\omega}}
% 	\left(
% 		\sqrt{n} \ket{\psi_{n-1}^{(0)}}
% 		-
% 		\sqrt{n+1} \ket{\psi_{n+1}^{(0)}}
% 	\right)\right].
% \end{align}
% \begin{align}
% 	\ket{n}^{(1)}
% 	&\nonumber=
% 	\sum_{m \neq n}
% 	\frac{\braket{m|^{(0)} \hat H | n}^{(0)}}{E_n^{(0)} - E_m^{(0)}}
% 	\ket{m}^{(0)}
% 	=
% 	\epsilon
% 	\sqrt{\frac{\hbar}{2m\omega}}
% 	\sum_{m \neq n}
% 	\frac{	
% 	\braket{m|^{(0)} \hat a^\dagger + \hat a | n}^{(0)}
% 	}{
% 	\hbar \omega(n - m)
% 	}
% 	\ket{m}^{(0)}
% 	\\
% 	&=\nonumber
% 	\epsilon
% 	\sqrt{\frac{\hbar}{2m\omega}}
% 	\left(
% 	\sum_{m \neq n}
% 	\frac{	
% 		\sqrt{n+1}
% 		\braket{m|^{(0)} |n+1}^{(0)}
% 	% \braket{m|^{(0)} \hat a^\dagger + \hat a | n}^{(0)}
% 	}{
% 	\hbar \omega(n - m)
% 	}
% 	\ket{m}^{(0)}
% 	+
% 	\sum_{m \neq n}
% 	\frac{	
% 		\sqrt{n}
% 		\braket{m|^{(0)} |n-1}^{(0)}
% 	% \braket{m|^{(0)} \hat a^\dagger + \hat a | n}^{(0)}
% 	}{
% 	\hbar \omega(n - m)
% 	}
% 	\ket{m}^{(0)}
% 	\right)
% 	\\
% 	&\nonumber=
% 	\epsilon
% 	\sqrt{\frac{\hbar}{2m\omega}}
% 	\left(
% 	\frac{\sqrt{n+1}
% 	% \braket{m|^{(0)} \hat a^\dagger + \hat a | n}^{(0)}
% 	}{
% 	\hbar \omega(n - (n+1))
% 	}
% 	\ket{n+1}^{(0)}
% 	+
% 	% \sum_{m \neq n}
% 	\frac{
% 		\sqrt{n}
% 	% \braket{m|^{(0)} \hat a^\dagger + \hat a | n}^{(0)}
% 	}{
% 	\hbar \omega(n - (n-1))
% 	}
% 	\ket{n-1}^{(0)}
% 	\right)
% 	\\
% 	&=
% 	\frac{\epsilon}{\sqrt{2 m \omega^3 \hbar}}
% 	\left(
% 		-\sqrt{n+1}\ket{n+1}^{(0)}
% 		+
% 		\sqrt{n}
% 		\ket{n-1}^{(0)}
% 	\right).
% 	% \epsilon
% 	% \sqrt{\frac{\hbar}{2m\omega}}
% \end{align}
The second-order corrections to the energies are computed very similarly to the first-order corrections to the stationary states:
\begin{align}
	E^{(2)}_n
	&=\nonumber
	\sum_{m\ne n}
	\frac{|\braket{\psi_m^{(0)} | \hat H' | \psi_n^{(0)}}|^2}{E^{(0)}_n - E_m^{(0)}}
	=
	\frac{\epsilon^2}{2m\omega^2}
	\sum_{m\ne n}
	\frac{|
	\braket{\psi_m^{(0)} | \hat x | \psi_n^{(0)}}
	|^2}{n-m}
	=
	\frac{\epsilon^2}{2m\omega^2}
	\sum_{m\ne n}
	\frac{|
	\braket{\psi_m^{(0)} | \hat a + \hat a^\dagger| \psi_n^{(0)}}
	|^2}{n-m}
	\\
	&
	=
	\frac{\epsilon^2}{2m\omega^2}
	\sum_{m\ne n}
	\frac{|
	\sqrt{n}
	\braket{\psi_m^{(0)} | \psi_{n-1}^{(0)}}
	+
	\sqrt{n+1}
	\braket{\psi_m^{(0)} | \psi_{n+1}^{(0)}}
	|^2}{n-m}
	=
	\frac{\epsilon^2}{2m\omega^2}
	\sum_{m\ne n}
	\frac{|
	\sqrt{n}
	\delta_{m,n-1}
	+
	\sqrt{n+1}
	\delta_{m,n+1}
	|^2}{n-m}
	\nonumber
	\\
	&=\nonumber
	\frac{\epsilon^2}{2m\omega^2}
	\left[
		0+\cdots+0
		+
		\frac{|
		\sqrt{n}
		\delta_{n-1,n-1}
		+
		\sqrt{n+1}
		\delta_{n-1,n+1}
		|^2}{n-(n-1)}
		+
		\frac{|
		\sqrt{n}
		\delta_{n+1,n-1}
		+
		\sqrt{n+1}
		\delta_{n+1,n+1}
		|^2}{n-(n+1)}
		+
		0+\cdots+0
	\right]
	\\
	&=
	\frac{\epsilon^2}{2m\omega^2}
	\left[
		|\sqrt{n}|^2
		-
		|\sqrt{n+1}|^2
	\right]
	=
	-\frac{\epsilon^2}{2m\omega^2}.
	\end{align}
	% \begin{align}
% 	E^{(2)}_n
% 	&=\nonumber
% 	\sum_{m \ne n}
% 	\frac{
% 		|\braket{m|^{(0)} \hat H' |n}^{(0)}|^2
% 	}{E_n^{(0)} - E_m^{(0)}}
% 	=
% 	% \frac{\epsilon^2}{2 m \omega^2}
% 	% \sum_{m \ne n}
% 	% \frac{|\bra{m}^{(0)} \hat a^\dagger + a \ket{n}^{(0)}|^2}{n -m}
% 	% =
% 	\frac{\epsilon^2}{2 m \omega^2}
% 	\sum_{m \ne n}
% 	\frac{
% 		|
% 		\bra{m}^{(0)} \hat a^\dagger\ket{n}^{(0)} +\bra{m}^{(0)} a \ket{n}^{(0)}
% 		|^2
% 	}{n -m}
% 	\\
% 	&=\nonumber
% 	\frac{\epsilon^2}{2 m \omega^2}
% 	\sum_{m \ne n}
% 	\frac{
% 		|
% 		\sqrt{n+1}
% 		\bra{m}^{(0)} \ket{n+1}^{(0)} +\sqrt{n}\bra{m}^{(0)} \ket{n-1}^{(0)}
% 		|^2
% 	}{n -m}
% 	% =
% 	% \frac{\epsilon^2}{2 m \omega^2}
% 	% \sum_{m \ne n}
% 	% \frac{
% 	% 	|
% 	% 	\sqrt{n+1}
% 	% 	\bra{m}^{(0)} \ket{n+1}^{(0)} +\sqrt{n}\bra{m}^{(0)} \ket{n-1}^{(0)}
% 	% 	|^2
% 	% }{n -m}
% 	\\
% 	&\nonumber
% 	=
% 	\frac{\epsilon^2}{2 m \omega^2}
% 	\left(
% 	% \sum_{m \ne n}
% 	\frac{
% 		|
% 		\sqrt{n+1}
% 		\bra{n-1}^{(0)} \ket{n+1}^{(0)} +\sqrt{n}\bra{n-1}^{(0)} \ket{n-1}^{(0)}
% 		|^2
% 	}{n -(n-1)}\right.
% 	\\
% 	&\quad +
% 	% \frac{
% 	% 	|
% 	% 	\sqrt{n+1}
% 	% 	\bra{n}^{(0)} \ket{n+1}^{(0)} +\sqrt{n}\bra{n}^{(0)} \ket{n-1}^{(0)}
% 	% 	|^2
% 	% }{n -n}
% 	% \\&\quad \nonumber
% 	% +
% 	\left.
% 	\frac{
% 		|
% 		\sqrt{n+1}
% 		\bra{n+1}^{(0)} \ket{n+1}^{(0)} +\sqrt{n}\bra{n+1}^{(0)} \ket{n-1}^{(0)}
% 		|^2
% 	}{n -(n+1)}
% 	\right)
% 	=
% 	\frac{\epsilon^2}{2m \omega^2}
% 	\left(
% 		n
% 		-
% 		(n+1)
% 	\right)
% 	=
% 	-
% 	\frac{\epsilon^2}{2m \omega^2}.
% \end{align}
To second order, the perturbed energies are,
\begin{align}
	E_n
	=
	\hbar \omega \left(n+\frac{1}{2}\right)
	-\frac{\epsilon^2}{2m \omega^2}
	+
	O(\epsilon^3).
\end{align}
% and to first order, the perturbed stationary states are,
% \begin{align}
% 	\ket{n}
% 	=
% 	\ket{n}^{(0)}
% 	+
% 	\frac{\epsilon}{\sqrt{2 m \omega^3 \hbar}}
% 	\left(
% 		-\sqrt{n+1}\ket{n+1}^{(0)}
% 		+
% 		\sqrt{n}
% 		\ket{n-1}^{(0)}
% 	\right)
% 	+
% 	O(\epsilon^2).
% \end{align}


\paragraph{Exact solution:}
Observe that if we shift the potential by $x_0$, we have,
\begin{align}
	\frac{1}{2} m \omega^2 ( x - x_0)^2
	=
	\frac{1}{2} m \omega^2 x^2
	-
	m \omega^2 x_0 x 
	+
	\frac{1}{2} m \omega^2 x_0^2,
\end{align}
so we can reproduce the perturbed potential using $\epsilon =- m \omega^2 x_0 \implies x_0 = -\epsilon/(m \omega^2)$ up to an overall constant:
\begin{align}
	\frac{1}{2} m \omega^2 \left(
		x + \frac{\epsilon}{m \omega^2}
	\right)^2
	=
	\frac{1}{2} m \omega^2 x^2
	+
	\epsilon x
	+
	\frac{\epsilon^2}{2 m \omega^2}.
\end{align}
As $\hat p$ is remains invariant under the transformation $x \to x -x_0$, we can write,
\begin{align}
	\hat H + \hat H'
	=
	\left.\hat H\right|_{x \to x + \epsilon/(m \omega^2)}
	-
	\frac{\epsilon^2}{2m \omega^2},
\end{align}
where $\left.\hat H\right|_{x \to x + \epsilon/(m \omega^2)}$ is the Hamiltonian with the $x$-coordinate shifted.
The stationary states of $\left.\hat H\right|_{x \to x + \epsilon/(m \omega^2)}$ are the stationary states of $\hat H$, shifted by $- \epsilon/(m \omega^2)$, and shifting the Hamiltonian by a constant does not change the eigenstates (up to a phase).
Therefore, the eigenstates of $\hat H + \hat H'$ are the usual QHO eigenfunctions, shifted by $-\epsilon/(m \omega^2)$, and the energy eigenvalues are those of the QHO, shifted by $-\epsilon^2/(2m \omega^2)$, so the energy eigenvalues are $E_n = 
\hbar \omega(n + \frac{1}{2}) - \epsilon^2/(2m\omega^2)$.
This shows that the perturbative expansion of the energy eigenvalues is exact to second order!


% \paragraph{A conceptual note:}

% We could have computed all these products using an integral like so:
% \begin{align}
% 	\ip{\psi_n | }
% \end{align}





\end{document}





